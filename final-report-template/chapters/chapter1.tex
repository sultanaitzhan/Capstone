% Chapter 1

\chapter{Introduction} % Main chapter title

\label{chapter1} % For referencing the chapter elsewhere, use \ref{Chapter1}

Often, mathematical modelling of the real-world phenomena results in ordinary and partial differential equations. Depending on equations, mathematicians may or may not have the tools to obtain solutions or understand the phenomenon very well. Fortunately, there are techniques that allow one to deal with differential equations without solving them. One such tool is asymptotic analysis, which leads to simplified equations that are very similar to original equations. As such, solutions of the simplified equations may model the real-world phenomenon, subject to some error estimates.

In particular, one problem that is amenable to asymptotic methods is the \textit{water waves problem}, which describes the behaviour of water and its surface under certain conditions. Assuming an irrotational, incompressible, and inviscid fluid in one dimension, and a domain with flat bottom, the equations of fluid motion are given by
\begin{subequations} \label{S1:DimWholeLineProblem}
\begin{align}
\phi_{xx} + \phi_{zz} &= 0 &-h < z < \eta(x,t) \label{S1:PDE}\\
\phi_{z} &= 0 &z = -h \label{S1:BBC}\\
\eta_t + \phi_{x}\eta_{x} &= \phi_{z} & z = \eta(x,t) \label{S1:KBC}\\
\phi_t + g\eta + \frac{1}{2}(\phi_{x}^2 + \phi_{z}^2) &= 0 &z = \eta(x,t) \label{S1:DBC}
\end{align}
\end{subequations}
where $\phi(x,z)$ is the fluid velocity and $\eta(x)$ is the surface elevation. In addition, $z$ is the vertical coordinate, $x$ is the horizontal direction, and $g$ is acceleration due to gravity. We let $x\in\RR,$ so that \eqref{S1:DimWholeLineProblem} is the water wave problem on the whole line. Although nonlinear partial differential equations (PDEs) \eqref{S1:KBC} and \eqref{S1:DBC} are hard to solve on their own, what makes the problem \eqref{S1:DimWholeLineProblem} truly difficult is the need to solve the Laplace's equation \eqref{S1:PDE} on a domain whose shape is unknown. 

To make the equations of motion more tractable, one can reformulate the problem and apply the tools of asymptotics. Of particular interest is the work \cite{AFM2006}, henceforth referred to as the AFM formulation. In this paper, authors rewrite \eqref{S1:DimWholeLineProblem} as a system of two equations, for the surface variable $q(x) = \phi(x, \eta(x)),$ i.e. the velocity evaluated at the surface. Taking advantage of a new system, various asymptotic reductions are performed. In well-understood physical conditions, the simplified equations correspond to the expected equations.

One interesting physical condition is the shallow water regime, which is defined by small-amplitude waves that have a small depth relative to the wavelength. In this regime, asymptotic tools reveal that the fluid behaviour is governed by the following approximate equations: the \textit{wave} equation
\begin{equation}\label{S1:eq1}
q_{tt} - q_{xx} = 0,
\end{equation} 
and two \textit{Korteweg de Vries (KdV)} equations
\begin{equation}\label{S1:eq2}
\begin{aligned}
F_{T} + \frac{1}{3}F_{\xi \xi \xi} + 3 F F_{\xi} &= 0, \\
G_{T} + \frac{1}{3}G_{\zeta \zeta \zeta} + 3 G G_{\zeta} &= 0, \\
\end{aligned}
\end{equation}
where $q(x,t) = F(x-t, T) + G(x+t, T).$ We refer to the model given by \eqref{S1:eq1} and \eqref{S1:eq2} as the \textit{KdV model}.

In this capstone project, we consider an alternative formulation of the problem \eqref{S1:DimWholeLineProblem}, as presented in \cite{OV2013}. Although slightly different from AFM formulation, it is contended  that this formulation is well-suited for performing asymptotics. We further advocate the efficacy of this formulation, by deriving the equations \eqref{S1:eq1} and \eqref{S1:eq2}. As a brief outline, in Chapter 2, we introduce the reader to asymptotic analysis, explain the physical assumptions of the problem, and briefly explain why the model \eqref{S1:eq1} and \eqref{S1:eq2} is a good approximation to the water wave problem. In Chapter 3, we derive the formulation, perform asymptotics, and derive the approximate equations. In Chapter 4, we justify the utility of the model, contrasting and comparing with other formulations. In Chapter 5, we describe an application of the formulation to a water wave problem on \textit{the half line}.
