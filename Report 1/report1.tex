% Preamble
\documentclass[10pt,reqno,oneside,a4paper]{article}
\usepackage[a4paper,includeheadfoot,left=25mm,right=25mm,top=00mm,bottom=20mm,headheight=20mm]{geometry}
\usepackage{siunitx}
% Standard packages
\usepackage{amssymb,amsmath,amsthm}
\usepackage{xcolor,graphicx}
\usepackage{verbatim}
\usepackage{hyperref}
% To use turkish characters
\usepackage[utf8]{inputenc}
% Layout of headers & footers
\usepackage{titling}
\usepackage{fancyhdr}
\pagestyle{fancy} \lhead{{\theauthor}} \chead{} \rhead{{\theshorttitle}} \lfoot{} \cfoot{\thepage} \rfoot{}

% Hyphenation
\hyphenation{non-zero}

% Theorem definitions in the amsthm standard
\newtheorem{thm}{Theorem}
\newtheorem{lem}[thm]{Lemma}
\newtheorem{sublem}[thm]{Sublemma}
\newtheorem{prop}[thm]{Proposition}
\newtheorem{cor}[thm]{Corollary}
\newtheorem{conc}[thm]{Conclusion}
\newtheorem{conj}[thm]{Conjecture}
\theoremstyle{definition}
\newtheorem{defn}[thm]{Definition}
\newtheorem{cond}[thm]{Condition}
\newtheorem{asm}[thm]{Assumption}
\newtheorem{ntn}[thm]{Notation}
\newtheorem{prob}[thm]{Problem}
\theoremstyle{remark}
\newtheorem{rmk}[thm]{Remark}
\newtheorem{eg}[thm]{Example}
\newtheorem*{hint}{Hint}

%% Mathmode shortcuts
% Number sets
\newcommand{\NN}{\mathbb N}              % The set of naturals
\newcommand{\NNzero}{\NN_0}              % The set of naturals including zero
\newcommand{\NNone}{\NN}                 % The set of naturals excluding zero
\newcommand{\ZZ}{\mathbb Z}              % The set of integers
\newcommand{\QQ}{\mathbb Q}              % The set of rationals
\newcommand{\RR}{\mathbb R}              % The set of reals
\newcommand{\CC}{\mathbb C}              % The set of complex numbers
\newcommand{\KK}{\mathbb K}              % An arbitrary field
% Modern typesetting for the real and imaginary parts of a complex number
\renewcommand{\Re}{\operatorname*{Re}} \renewcommand{\Im}{\operatorname*{Im}}
% Upright d for derivatives
\newcommand{\D}{\ensuremath{\,\mathrm{d}}}

\newcommand{\X}{\ensuremath{{\bf x}}}

\newcommand{\U}{\ensuremath{{\bf u}}}

\newcommand{\N}{\ensuremath{{\bf n}}}

\newcommand{\XX}{\ensuremath{{\bf \xi}}}

% Upright i for imaginary unit
\newcommand{\ri}{\ensuremath{\mathrm{i}}}
% Upright e for exponentials
\newcommand{\re}{\ensuremath{\mathrm{e}}}
% abbreviation for \lambda
\newcommand{\la}{\ensuremath{\lambda}}
% Make epsilons look more different from the element symbol
\renewcommand{\epsilon}{\varepsilon}
% Always use slanted forms of \leq, \geq
\renewcommand{\geq}{\geqslant}
\renewcommand{\leq}{\leqslant}
% Shorthand for "if and only if" symbol
\newcommand{\Iff}{\ensuremath{\Leftrightarrow}}
% Make bold symbols for vectors
\providecommand{\BVec}[1]{\mathbf{#1}}
% Hyperbolic functions
\providecommand{\sech}{\operatorname{sech}}
\providecommand{\csch}{\operatorname{csch}}
\providecommand{\ctnh}{\operatorname{ctnh}}
% sinc function
\providecommand{\sinc}{\operatorname{sinc}}

% add two sub and superscripts with a space between them
\newcommand{\Mspacer}{\;} %Spacer for below Matrix display functions
\newcommand{\M}[3]{#1_{#2\Mspacer#3}} %Print a symbol with two subscripts eg a matrix entry
\newcommand{\Msup}[4]{#1_{#2\Mspacer#3}^{#4}} %Print a symbol with two subscripts and a superscript eg a matrix entry
\newcommand{\Msups}[5]{#1_{#2\Mspacer#3}^{#4\Mspacer#5}} %Print a symbol with two subscripts and two superscripts eg a matrix entry
\newcommand{\MAll}[7]{\prescript{#1}{#2}{#3}_{#4\Mspacer#5}^{#6\Mspacer#7}} %Print a symbol with two subscripts and two superscripts eg a matrix entry

% Make really wide hat for Fourier transforms applied to large functions
\usepackage{scalerel}
\usepackage{stackengine}
\stackMath
\newcommand\reallywidecheck[1]{%
\savestack{\tmpbox}{\stretchto{%
  \scaleto{%
    \scalerel*[\widthof{\ensuremath{#1}}]{\kern-.6pt\bigwedge\kern-.6pt}%
    {\rule[-\textheight/2]{1ex}{\textheight}}%WIDTH-LIMITED BIG WEDGE
  }{\textheight}% 
}{0.5ex}}%
\stackon[1pt]{#1}{\scalebox{-1}{\tmpbox}}%
}
\providecommand{\widecheck}{\reallywidecheck}

\newcommand\reallywidehat[1]{%
\savestack{\tmpbox}{\stretchto{%
  \scaleto{%
    \scalerel*[\widthof{\ensuremath{#1}}]{\kern-.6pt\bigwedge\kern-.6pt}%
    {\rule[-\textheight/2]{1ex}{\textheight}}%WIDTH-LIMITED BIG WEDGE
  }{\textheight}% 
}{0.5ex}}%
\stackon[1pt]{#1}{\tmpbox}%
}
\author{Student: Sultan Aitzhan \\Supervisor: Prof. Katie Oliveras \\ Co-Supervisor: Prof. Dave Smith}
\title{Approximate expansions for wave \& KdV equations via the velocity potential and non-local formulations.}
\newcommand{\theshorttitle}{Report 1}
\newcommand{\theauthorr}{Sultan Aitzhan}
\date{\today}
\allowdisplaybreaks

\begin{document}
\maketitle
\thispagestyle{fancy}
\tableofcontents

\section{Introduction}
Often times, physical problems present one with systems of equations that are difficult to solve analytically. So, instead of solving such systems, one could perform an approximation procedure to obtain solutions. While they do not solve the system in the usual sense, such solutions still incorporate many features of the problem that gives an insight into the problem one wishes to investigate. In this report, we consider a specific system, termed a water wave problem. We study a particular phenomenon, namely long waves in shallow water, and approximate solutions of this problem in the leading order. 

As for the outline, we first give a mathematical description of the problem. Then, we approximate the solution of the problem on a whole line. Finally, we introduce the half-line problem, and perform a preliminary approximation procedure. We conclude with future directions of the project.

\subsection{The Euler's equation}
The relevant equations of fluid mechanics come from two core principles: conservation of mass, and conservation of momentum, and are given by:
\begin{align}
\frac{\partial \rho}{\partial t} + \nabla \cdot (\rho \V) &= 0, \label{CoMass} \\
\rho\left[ v\frac{\partial \V}{\partial t} + (\V \cdot \nabla) \V\right] &= \textbf{F} - \nabla P + v_{\star} \Delta \V. \label{CoMomentum}
\end{align}
In equations, $\rho = \rho(\X, t)$ denotes the fluid mass density, $\V = \V(\X,t)$ is the fluid velocity, $P$ refers to pressure, $\textbf{F}$ is an external force, and $v_{\star}$ is the kinematic viscosity due to frictional forces. Derivations of \eqref{CoMass} and \eqref{CoMomentum} can be found in many books, for example, see \cite[Chapter 3]{batchelor} or \cite[Chapter 1]{marsden}.

We'd like to derive equations that describe water waves from \eqref{CoMass} and \eqref{CoMomentum}, in a domain with a free surface. First, we assume that the mass density is constant ($\rho = \rho_0),$ and that the fluid is inviscid ($v_{\star} = 0$). These assumptions make sense: if water waves consist only of water, then the mass density is the same, and physically, water rarely resists changes in its shape. Then, the equations become:
\begin{align}
\nabla \cdot (\V) &= 0, \label{CoMass1} \\
\rho_0\left[ \frac{\partial \V}{\partial t} + (\V \cdot \nabla) \V\right] &= \textbf{F} - \nabla P. \label{CoMomentum1}
\end{align}
Furthermore, we suppose that water waves exhibit no rotation, which means that the curl of the velocity field vanishes, i.e. $\nabla \times \V = \vec{0}.$ This means that we can write $\U = \nabla \phi,$ for some scalar field $\phi.$ Then, \eqref{CoMass1} transforms into:
\begin{equation}
\Delta \phi = 0, \label{CoMass2} 
\end{equation}
and $\eqref{CoMomentum1}$ into:
\begin{align}
\rho_0\left[\frac{\partial \V}{\partial t} + (\V \cdot \nabla) \V \right] = \mathbf{F} -\nabla P &\implies \frac{\partial \V}{\partial t} + (\V \cdot \nabla) \V = - \nabla \left( \frac{P+U}{\rho_0}\right) \nonumber\\
&\implies \frac{\partial \V}{\partial t} + \frac{1}{2}\nabla ( \V \cdot \V) - \V \times (\nabla \times \V) = - \nabla \left( \frac{P+U}{\rho_0}\right) \nonumber\\
&\implies \frac{\partial \V}{\partial t} + \nabla \left(\frac{1}{2} \V \cdot \V + \frac{P+U}{\rho_0}\right) = \V \times (\nabla \times \V) \nonumber\\
&\implies \frac{\partial \V}{\partial t} + \nabla \left(\frac{1}{2} \Vert \V \Vert^2 + \frac{P+U}{\rho_0}\right) = \vec{0}  \nonumber\\
&\implies \nabla \left(\frac{\partial \phi}{\partial t} + \frac{1}{2} \Vert \V \Vert^2 + \frac{P+U}{\rho_0} \right) = \vec{0}  \nonumber\\
&\implies \frac{\partial \phi}{\partial t} + \frac{1}{2} \Vert \V \Vert^2 + \frac{P+U}{\rho_0} = f \label{CoMomentum2} 
\end{align}
where we let $\textbf{F} = -\nabla U$ for some scalar field $U.$ Finally, since $\V = \nabla \phi,$ we can let 
\[ 
\phi \mapsto \phi + \int^t_0 f(t') \D t',
\]
which yields 
\[ 
\frac{\partial \phi}{\partial t} + \frac{1}{2} \Vert \V \Vert^2 + \frac{P+U}{\rho_0} = 0.
\]

Thus, we have reduced the fluid equations to the water wave equations. We now consider the boundary conditions of the problem. Although the problem clearly makes sense physically, how do we know if it makes sense mathematically?

\subsubsection{Water wave problem in the velocity potential}
We now derive the relevant water wave problem. Conservation of mass, which, physically, determines the behaviour of a water wave inside the domain, is given by:
\[
\Delta \phi = 0 \qquad -h <z < \eta(x,y,t)
\]
Conservation of momentum is a statement about how forces of atmosphere affect the behaviour of a water wave at its surface. Note that this behaviour is distinct from the behaviour inside the domain, which is why $\phi$ and $\eta$ are separate quantities. Neglecting the effects of surface tension, we suppose that the dominant force is that of buoyancy, i.e. $\textbf{F} = - \nabla(\rho_0 gz),$ so that $U = \rho_0 g z,$ where $g$ is the gravitational constant of acceleration. Further, suppose that pressure vanishes at the surface, so that $P = 0.$ Thus, we have 
\[ 
\frac{\partial \phi}{\partial t} + \frac{1}{2} \Vert \nabla \phi \Vert^2 + g \eta = 0 \qquad z = \eta(x,y,t)
\]
Since this condition is about momentum, we term it as the \textit{dynamic} condition. 
Physically, we need one more condition at the surface: we require that the surface $\eta$ is a surface of the water wave, i.e. the surface $\eta$ is always composed of fluid particles. This is a geometric condition, for it deals with the shape of the surface. Thus, we require that the surface $F = z - \eta(x,y,t) = 0 $ for all times and positions, which mathematically can be written using the notion of a material derivative:
\[ 
\frac{DF}{Dt} = \frac{\partial F}{\partial t} + \V \cdot \nabla F = 0 \implies \frac{Dz}{Dt} = \frac{D\eta}{Dt}  \implies \frac{\partial \phi}{\partial z} =  \frac{\partial \eta}{\partial t} + \V \cdot \nabla \eta \qquad z = \eta(x,y,t),
\]
which follows since \[ \dfrac{Dz}{Dt} = \dfrac{\partial z}{\partial t} + \V \cdot \nabla z = \nabla \phi \cdot \nabla z = (\phi_x, \phi_y, \phi_z) \cdot (0,0,1) = \phi_z. \]
This condition is also known as the \textit{kinematic} condition. 

Finally, akin to the kinematic condition, we'd like to prescribe a geometric condition at the bottom. We assume that the bottom surface is impermeable, which can be expressed via material derivative. Thus, if $z = b(x,y,t) = -h$ is the bottom surface, we must have:
\[ 
F = z - b(x,y,t) \implies \frac{DF}{Dt} = 0 \implies \frac{Dz}{Dt} = \frac{D b}{Dt} \implies \frac{\partial \phi}{\partial z} = \frac{\partial b}{\partial t} + \V \cdot \nabla b = 0, \qquad z = b(x,y,t)= - h.
\]
It's worth pointing out that the absence of viscosity suggests that the bottom topography becomes a surface of the fluid, so that the fluid particles in contact with the bed move in this surface. As such, this condition somewhat mirrors the kinematic condition at a free surface, the notable difference being that the bottom is prescribed a priori. 

Lastly, we assume that the fluid is in equilibrium as $|x|, |y| \to \infty.$ To conclude, a water wave problem is given by the following equations:
\begin{subequations}\label{WLP2D}
\begin{align}
\label{2PDE} \Delta \phi &= 0  &-h <&z < \eta(x,y,t) \\
\label{2BC1} \phi_z &= 0 &z &= -h  \\ 
\label{2BC2} \phi_t + \frac{1}{2} \Vert \nabla \phi \Vert^2 + g \eta &= 0 &z &= \eta(x,y,t)\\
\label{2BC3} \eta_t + \nabla \phi \cdot \nabla \eta &= \phi_z &z &= \eta(x,y,t)
\end{align}
\end{subequations}
For simplicity, we consider a system that is ``flat'', i.e. there's no $y$-component. Equivalently, we are making an assumption that waves are only in one direction, and that there are no transverse waves. Further, suppose that water flows everywhere, i.e. on the whole plane in $x$. Then, along with a condition $|\phi| < \infty$ as $|x| \to \infty,$ \eqref{WLP2D} turns into 
\begin{subequations}\label{WLP1D}
\begin{align}
\label{1PDE} \phi_{xx} + \phi_{zz} &= 0  &-h <&z < \eta(x,t) \\
\label{1BC1} \phi_z &= 0 &z &= -h  \\ 
\label{1BC2} \phi_t + \frac{1}{2} (\phi_{x}^2 + \phi_{z}^2) + g \eta &= 0 &z &= \eta(x,t)\\
\label{1BC3}  \eta_t + \phi_x\eta_x &=\phi_z &z &= \eta(x,t)
\end{align}
\end{subequations}
where the last equation follows since $\nabla \phi \cdot \nabla \eta = (\phi_x, \phi_z) \cdot (\eta_x, 0) = \phi_x\eta_x.$ The problem \eqref{WLP1D} is known as a \textbf{water wave problem} on the whole line, where the whole line comes from the fact that $x$ is any real number. Although non-linear partial differential equations (PDEs) \eqref{1BC2} and \eqref{1BC3} are hard to solve on their own, what makes the problem \eqref{WLP1D} truly difficult is that we are trying to solve the Laplace's equation \eqref{1PDE} on a domain whose shape we do not even know!

\rmk{The reader should be aware that surface tension is made negligible in \eqref{WLP1D}. The extension of the dynamic condition \eqref{1BC2} that accommodates the effects of surface tension is 
\[ 
\phi_t + \frac{1}{2} (\phi_x^2 + \phi_z^2) + g\eta = \frac{\sigma}{\rho_0} \frac{\eta_{xx}}{(1+\eta_x^2)^{3/2}} \qquad z = \eta(x, t),
\]
where $\sigma$ is the coefficient of surface tension.}
\rmk{The way we expressed the water wave problem is in terms of the scalar field $\phi.$ Now, recall that $\V = \nabla \phi,$ where $\V$ is the velocity field of the fluid. Since $\phi$ is a potential of $\V,$ we refer to $\phi$ as the \textit{velocity potential}, and the formulations \eqref{WLP2D} and \eqref{WLP1D} are called the velocity potential formulation of the water wave problem. There is also an integral formulation of the problem, with which we will deal later.}

\subsubsection{Dispersion relation}

\subsubsection{Nondimensionalisation}
Having expressed the problem in the velocity potential, we'd like to remove the dimensional variables. Since dimensions of the problem are directly related to the units of variables (wavelength, time, height), it is hard to decide which terms are negligible when performing an approximation procedure. Thus, we'd like to remove the dimensions of the problem, and work with ``pure'' numbers.  Define new dimensionless variables as follows:
\[ 
z = h z' \qquad x = \lambda_x x' \qquad t = \frac{\lambda_x}{c_0} t' \qquad \eta = a \eta' \qquad \phi  = \frac{\lambda_x ga}{c_0} \phi',
\]
where $c_0 = \sqrt{gh}$ is the speed of shallow water waves, $\lambda_x$ is a typical wavelength of the initial data, and $a$ is the maximum amplitude of the initial data. Note that primed variables are dimensionless. We transform the problem \eqref{WLP1D}.
First, by chain rule, we have
\[ 
\phi_{xx} = \frac{ga}{\lambda_x c_0} \phi_{x'x'}' \qquad \phi_{zz} = \frac{\lambda_x ga}{h^2c_0} \phi_{z'z'}'\] 
so that the PDE \eqref{1PDE} becomes
\[\frac{ga}{\lambda_x c_0} \phi_{x'x'}' + \frac{\lambda_x ga}{h^2c_0} \phi_{z'z'}' = 0 \implies \frac{1}{\lambda_x^2} \phi_{x'x'}' + \frac{1}{h^2} \phi_{z'z'}' = 0.
\]
The interval $z \in (h, \eta)$ becomes $hz' \in [-h, a h \eta'],$ so that $z' \in [-1, \frac{a \eta'}{h}].$ For the bottom condition \eqref{1BC1}, we have
\[ 
\phi_z =  \frac{\lambda_x ga}{h c_0} \phi_{z'}',
\]
so that 
\[ \phi_z = 0 \qquad z = - h \implies \phi_{z'}' = 0 \qquad z = -1. \]
Now, note that 
\[ 
\phi_t = ga \phi_{t'}' \qquad \phi_x^2 = \left(\frac{ga}{c_0} \phi_{x'}'\right)^2 = \left(\frac{ga}{c_0}\right)^2 \left(\phi_{x'}'\right)^2 \qquad \phi_z^2 = \left(\frac{\lambda_x ga}{hc_0}\phi_{z'}'\right)^2 = \frac{\lambda_x^2}{h^2}\left(\frac{ga}{c_0}\right)^2\left(\phi_{z'}'\right)^2 \qquad g\eta = ga\eta'
\]
so that the dynamic condition \eqref{1BC2} transforms into:
\begin{align*}
\phi_t + \frac{1}{2} (\phi_{x}^2 + \phi_{z}^2) + g \eta = 0 &\implies ga \phi_{t'}' + \frac{1}{2}\left(\frac{ga}{c_0}\right)^2 \left((\phi_{x'}')^2 + \frac{\lambda_x^2}{h^2}(\phi_{z'}')^2\right) + ga \eta' = 0 \\
&\implies \phi_{t'}' + \frac{a}{2h} \left((\phi_{x'}')^2 + \frac{\lambda_x^2}{h^2}(\phi_{z'}')^2\right) + \eta' = 0,
\end{align*}
at $z' = a\eta'/h.$ Finally, note that 
\[ 
\eta_t = \frac{ac_0}{\lambda_x}\eta_{t'}' \qquad \eta_x = \frac{a}{\lambda_x}\eta_{x'}',
\]
so that the kinematic condition \eqref{1BC3} becomes
\begin{align*}
\phi_z = \eta_t + \phi_x\eta_x &\implies \frac{\lambda_x ga}{h c_0} \phi_{z'}'=\frac{ac_0}{\lambda_x}\eta_{t'}' + \frac{ga^2}{\lambda_xc_0} \phi_{x'}' \eta_{x'}' \\
&\implies \frac{\lambda_x^2}{h^2} \phi_{z'}'= \eta_{t'}' + \frac{a}{h} \phi_{x'}' \eta_{x'}',
\end{align*}
at $z' = a\eta'/h.$ Now, define dimensionless parameters $\varepsilon = a/h$ and $\mu = h/\lambda_x;$ physically, $\varepsilon$ is a measure of amplitude of the wave, and $\mu$ is a measure of the depth relative to the typical wavelength. Alternatively, $\varepsilon$ is a measure of nonlinearity, and $\mu$ is a measure of dispersion. Bringing all together and dropping the primed notation, we obtain a non-dimensionalised problem:
\begin{subequations}\label{WLP1DND1}
\begin{align}
\label{1PDEND1}  \mu^2 \phi_{xx} + \phi_{zz} &= 0 &-1 <&z < \varepsilon\eta \\
\label{1BC1ND1} \phi_z &= 0 &z &= -1  \\ 
\label{1BC2ND1} \phi_{t} + \frac{\varepsilon}{2} \left(\phi_{x}^2 + \frac{1}{\mu^2}\phi_{z}^2\right) + \eta &= 0 &z &= \varepsilon\eta(x,t)\\
\label{1BC3ND1} \mu^2 \left[\eta_{t} + \varepsilon \phi_{x} \eta_{x}\right] &= \phi_{z} &z &= \varepsilon\eta(x,t).
\end{align}
\end{subequations}
with the condition that $|\phi| < \infty$ as $|x|\to \infty.$

\section{The whole-line problem}

In this section, we derive wave and Korteweg de Vries (KdV) equations on the whole line. First, we make assumptions about the relations between $\varepsilon$ and $\mu.$ We consider long waves in shallow water, which means that the depth $h$ is small relative to the wave wavelength $\lambda_x,$ i.e.
\[ \mu = \dfrac{h}{\la_x} \ll 1.\]
Further, suppose that waves have small amplitude, so 
\[ \varepsilon = \frac{a}{h} \ll 1. \]
Now, by Kruskal's principle of maximal balance, to obtain equations that are interesting, we should balance all of these assumptions by connecting them to each other. For this derivation, we choose $\varepsilon = \mu^2.$ This is to reflect the balance of ``weak nonlinearity'' and ``weak dispersion''. However, there is no reason not to balance in other ways, say $\varepsilon = \sqrt{\mu}.$ There are many options, and some of them will give interesting equations, while others will not lead to anything. As such, it is this assumption in our procedure that determines the relevance of to-be-derived equations. Thus, the nondimensional problem \eqref{WLP1DND1} becomes:
\begin{subequations}\label{WLP1DND2}
\begin{align}
\label{1PDEND2}  \varepsilon\phi_{xx} + \phi_{zz} &= 0 &-1 <&z < \varepsilon\eta \\
\label{1BC1ND2} \phi_z &= 0 &z &= -1  \\ 
\label{1BC2ND2} \phi_{t} + \frac{1}{2} \left(\varepsilon\phi_{x}^2 + \phi_{z}^2\right) + \eta &= 0 &z &= \varepsilon\eta(x,t)\\
\label{1BC3ND2} \varepsilon\left[\eta_{t} + \varepsilon \phi_{x} \eta_{x}\right] &= \phi_{z} &z &= \varepsilon\eta(x,t).
\end{align}
\end{subequations}
with the condition that $|\phi| < \infty$ as $|x|\to \infty.$

\subsection{Derivation of Wave \& KdV equations}

\subsubsection{Via velocity potential formulation}
We are now ready to perform an approximation procedure. 
\section{The half-line problem}

\subsection{Derivation of Wave \& KdV equations}

\subsubsection{Via velocity potential formulation}

\section{Future directions}
% \clearpage
\bibliographystyle{amsplain}
{\small\bibliography{references}}

\end{document}
