\documentclass[a4paper,reqno]{article}
\usepackage{amsmath,amssymb}
\usepackage[margin=1in]{geometry}
\usepackage{hyperref}

\title{Meeting Notes - Preparation for Semester 1 Capstone}
\date{Last Updated: \today}

\begin{document}
\maketitle
\section{Action Items}
 This section will contain a list of tasks assigned from the initial meeting. Items may be referenced in more detail within the following sections of the document.  
    \begin{itemize}
        \item \textbf{Agenda} 
            \begin{enumerate}
            	   \item Go through previous tasks.
                \item Project Proposal due Sep 16.
                \item Funding Application, due Sep 18.
                \item Linearization of half-line equations.
                \item Perturbation expansions.
                \item Next meeting.
            \end{enumerate}
        \item \textbf{From Previous Meeting}
            \begin{enumerate}
                
                \item \emph{Derivation of Euler}. Read through Sections 1.1, 1.2, and 2.1 from \cite{chorin1990mathematical}.   
                \item \emph{Derivation of Euler}. Read Chapter 1 of \cite{johnson1997modern}.
                \item \emph{Euler for Irrotational Flow}.  It's important to note that many people formulation the free-surface problem in terms of $u(x,z,t)$, $v(x,z,t)$, $p(x,z,t)$, and $\eta(x,t)$.  That is, in terms of the bulk velocities, the pressure, and the free surface.  However, in \eqref{eqn:oEuler1}-\eqref{eqn:oEuler4}, the problem is formulated in terms of a velocity potential $\phi(x,z,t)$ and $\eta(x,t)$.  Section 1.1.3 from \cite{johnson1997modern} has a good discussion of this difference (rotational vs. irrotational).
                \item Compare derivations and information from both \cite{chorin1990mathematical,johnson1997modern}.
                \item Choose either the scaling in Section 4.1 from \cite{deconinckNotes} or those given in Section 1.3 of \cite{johnson1997modern}.  Your choice (note the differences).  Nondimensionalize Euler's equations in the velocity potential formulation.
                \item Derive the wave equation from the Euler's equations.  You may find the discussion in Section 4.1.4 from \cite{deconinckNotes} useful.
                \item \emph{Background knowledge.} Watch these videos \href{https://www.youtube.com/playlist?list=PLfF--3o8i4r82vJ0kjCVYgqKgyVM5QwN0}{YouTube MIT playlist}
            \end{enumerate}
 
\item \textbf{Agenda for the next meeting/Post Meeting Notes}
 \begin{enumerate}
            	   \item Project Proposal: 
            	   \newline {\bf KL} to add information to Exploration subsection of Challenges, about what can be done further (how does our method compare to other methods, is this a better model, how useful in practice, student may develop his own questions, etc.), add the grading rubric; 
            	   \newline
            	  \newline {\bf SA} to complete Subject Areas section, add a sentence about the exploration to Scope section (smth like ``the third goal, depending on results, is to examine the application and utility of results"), factually correct the text written in Scope section, adjust the timeline (allocate one week to the reduction to the single equation).
                \item Funding Proposal
            	   \newline {\bf KL} to ping her colleagues about the SIAM conference;  
            	   \newline
            	  \newline {\bf SA} to complete the application form, ask DoS people about the invitation letter (bc the registration hasn't started)
                \item Next meeting on Wed, Sep 11, 8-9am, SGT.
                \item {\bf SA} to obtain Ipad (or any other equivalent device) for the next next meeting, to familiarize with the derivation of KdV equation.
            \end{enumerate}
               \end{itemize}
\bibliographystyle{plain}
\bibliography{references.bib}

\end{document}