\documentclass[10pt, oneside, a4paper]{article}


%--------This section sets up the document class and packages.
\usepackage[includeheadfoot, margin=20mm, headheight=20mm]{geometry} 
\usepackage{fancyhdr}
\usepackage{mwe}
\usepackage{amsmath}
\usepackage{amsthm}
\usepackage[utf8]{inputenc}
\usepackage{amssymb}
\usepackage{xcolor,graphicx}
\usepackage{hyperref}
\usepackage{centernot}
\usepackage{tikz}  % Uncomment this line iff you are using the tikz package to add drawings
\usepackage{tkz-euclide}
\usepackage{pgf}
\usepackage{pgfplots}
\pgfplotsset{compat=newest}
\pgfplotsset{plot coordinates/math parser=false}
\usepackage[toc,page]{appendix}
\usepackage{pgfgantt}

\allowdisplaybreaks
\pgfplotsset{soldot/.style={color=blue,only marks,mark=*}} \pgfplotsset{holdot/.style={color=blue,fill=white,only marks,mark=*}}

\usepackage{tocloft}
\renewcommand{\cftsecleader}{\cftdotfill{\cftdotsep}}
\renewcommand\labelenumi{(\roman{enumi})}

%--- This section makes possible customized theorem numbering.
\newtheorem{innercustomgeneric}{\customgenericname}
\providecommand{\customgenericname}{}
\newcommand{\newcustomtheorem}[2]{%
  \newenvironment{#1}[1]
  {%
   \renewcommand\customgenericname{#2}%
   \renewcommand\theinnercustomgeneric{##1}%
   \innercustomgeneric
  }
  {\endinnercustomgeneric}
}
\newcustomtheorem{cthm}{Theorem}
\newcustomtheorem{caxm}{Axiom}
\newcustomtheorem{clem}{Lemma}
\newcustomtheorem{ccor}{Corollary}
\newcustomtheorem{cprop}{Proposition}
\newcustomtheorem{cdefn}{Definition}
\newcustomtheorem{ceg}{Example}
\newcustomtheorem{crmk}{Remark}
\newcustomtheorem{ccus}{}
\newcustomtheorem{cprob}{Problem}

%---The following code sets up the way theorems are typeset and labeled.
%\newtheorem{thm}{Theorem}
\newtheorem*{thm}{Theorem}
%\newtheorem{lem}[thm]{Lemma}
\newtheorem*{lem}{Lemma}
%\newtheorem{cor}[thm]{Corollary}
\newtheorem*{cor}{Corollary}
%\newtheorem{prop}[thm]{Proposition}
\newtheorem*{prop}{Proposition}
\theoremstyle{definition}
%\newtheorem{defn}[thm]{Definition}
\newtheorem*{defn}{Definition}
\newtheorem*{axm}{Axiom}
\newtheorem*{eg}{Example}
\theoremstyle{remark}
%\newtheorem{rmk}[thm]{Remark}
\newtheorem*{rmk}{Remark}
\newtheorem*{sol}{Solution}

%---The following code defines a few extra commands that will be useful in some exercises.
\newcommand{\abs}[1]{\lvert#1\rvert}     % Absolute value symbol
\newcommand{\Abs}[1]{\Bigg\lvert#1\Bigg\rvert}     % Absolute value symbol (big)
\newcommand{\Z}{\mathbb Z}              % The set of integers
\newcommand{\Q}{\mathbb Q}              % The set of rationals
\newcommand{\R}{\mathbb R}              % The set of reals
\newcommand{\N}{\mathbb N}              % The set of natural numbers
\newcommand{\C}{\mathbb C}              % The set of complex numbers
\newcommand{\F}{\mathbb F}  

\newcommand{\ZZ}{\mathcal{Z}}
\newcommand{\OO}{\mathcal{O}}
\newcommand{\CC}{\mathcal{C}}
\newcommand{\UU}{\mathcal{U}}
\newcommand{\power}{\mathcal{P}}         % The power set of a set
\newcommand{\bfun}{\mathcal{F}}          % The finite subsets
\newcommand{\Id}{\mathrm{Id}}            % The identity function
\newcommand{\nil}{\emptyset}             % Empty set
\newcommand{\inflim}[1]{\lim_{#1\to\infty}} % Limit to infinity
\newcommand{\ninflim}[1]{\lim_{#1\to\infty}}% Limit to negative infinity
\providecommand{\BVec}[1]{\mathbf{#1}}   % Bold font for vectors
\DeclareMathOperator{\gon}{gon}          % A polygon
\DeclareMathOperator{\Fun}{Fun}          % The set of all functions from one set to another
\DeclareMathOperator{\Perm}{Perm}        % The set of all permutations on a set
\DeclareMathOperator{\Int}{int} %interior of a set
\DeclareMathOperator{\cl}{cl} %closure of a set
\DeclareMathOperator{\diam}{diam}
\DeclareMathOperator{\sinc}{sinc}
%\newcommand{\Re}{\mathrm{Re}} %real part
%---header/style/enumeration-------
\pagestyle{fancy}
\lhead{Capstone}
\chead{MCS}
\rhead{2019-2020 Semester 1}
\author{
    Student: Sultan Aitzhan\\
    Supervisor: Professor Katie Oliveras
    }
%----------------------------------

%-----------MY INFORMATION---------
%\title{{\fontfamily{qbk}\selectfont
\title{\textsc{Capstone Proposal}}
\date{\vspace{-5ex}} 
%----------------------------------
%unnumbered sections in TOC
\setcounter{secnumdepth}{0}

%-----This is where the GOOD STUFF begins---
\begin{document}

\maketitle

\thispagestyle{fancy}


\section{Title}
Approximate expansions for Wave \& KdV equations via the velocity potential and non-local formulations.

\section{Subject Areas}
Partial differential equations, asymptotic analysis, fluid dynamics, nonlinear waves, numerical analysis, integral equations.

\section{Challenges}
\subsection{Theoretical knowledge}
The approximation procedure for the Euler Equations in the two formulations requires a solid understanding of the Euler's equations and the non-local formulation. The student has taken advanced courses such as OPDE and Numerical Analysis. The student will further familiarise himself with the specific topics such as perturbation series, non-local formulation, and Euler's equations over the course of semester 1 and semester 2 as needed.

\subsection{Exploration}
As outlined in the next section, the primary purpose of this project is to systematically derive asymptotic shallow-water models for water-waves when boundary conditions are posed on only one side of the domain.  To the best of our knowledge, this will be the first systematic derivation and constitutes new research. 

As in the nature of mathematical research, there are many different avenues for exploration that can be pursued according to expressed interest.  These projects include, new asymptotic model equations, comparisons between models in terms of both analytic and computational issues, as well as potential experimental validation. In other words, we hope to derive new model equations that are rigorously derived, capture the correct physical phenomena, and are easy to analyze as well as simulate.

\section{Scope}
Although the wave \& KdV equations have been derived from the Euler's equations on the whole line, no corresponding work has been done at least in the case of the KdV equation on the half-line. In addition, it was recently shown that the Euler's equations can be a reduced to one, time-dependent equation. However, the wave \& KdV equations have yet to be derived from the single, time-dependent equation, for both a whole line, and a half line.

The first goal of this project is to derive the wave \& KdV equations from the Euler's equations on a half-line, while determining an appropriate boundary condition. The second goal is to derive the wave \& KdV equations from the single, time-dependent equation on the whole line and a half-line. Finally, depending on results, the third goal of the project is to examine the utility and potential applications of results, which will be accomplished via exploration. 

\section{Expectations associated with grade achievement}
As is the nature with research, it is possible that the project may not be completed in the time allocated.  However, the expectation is that the student will learn how to rigorously derive asymptotic models from Euler's equations for water waves, and understand when they can be used appropriately.  

In order to obtain an A- grade, the advisor has the following expectations in terms of mathematical content in the final written report:
\begin{itemize}
    \item A clear mathematical introduction to the "water-wave" problem as described in the ``Initial Capstone Meeting Notes'' (\emph{dated 10 August, 2019}).
    \item A rigorous derivation of the wave equation and the KdV equation on the whole line.  This includes a careful description of the scales used. This derivation should be completed using the two separate formulations outlined in the above section.  It is worth noting that the first derivation will simply be repeating a classical result, while the second will be a new derivation.
    \item A systematic numerical comparison between solutions to the wave equation and the KdV equation on the whole line. This includes a discussion of the assumptions made in the numerical algorithm, the associated numerical error, and how to control it.
    \item A detailed discussion of the proper formulation of the water-wave problem on the half-line including the appropriate boundary conditions and either the derived half-line models, or a detailed discussion of the obstacles faced when deriving the models.  
\end{itemize} 

The above should be written clearly and precisely in the student's own words.  The writing should demonstrate a clear understanding of the background materials as well as how they are synthesized together to solve this problem.  The organizational structure of the paper should be high quality on many levels including well-organized and motivated calculations/derivations, and sections that build on or complement one another.  The mathematical notation should be used properly.  


\section{Monthly plan with time allocation}

Estimated consultation time is one hour per week. Consultation consists of one-on-one video meetings.

\begin{center}
    \begin{tabular}{|l|l|l|} 
        \hline
        Time & Task & Deliverable\\
        \hline
        August & Derive Euler's equations in the velocity potential \& nondimensionalise the equations.  & \\ 
        \hline
        September & Derive the wave \& KdV equations on a whole line & \\ 
        \hline
        16 Sep 5pm & & Proposal \\
        \hline
        October &  Derive wave \& KdV equations on a half line &\\
        \hline
        November & Reduce Euler's equations to a single equation on a whole line and a half-line &\\
        \hline 
        15 Nov 5pm & & Report 1 \\
        \hline
        December & Exploration &\\
        \hline
        January & Exploration &\\
        \hline
        February & Exploration. &\\
        \hline
        March & Write-up the results &\\
        \hline
        April & Submission and Presentation &\\
        \hline 
        April 3 5pm & Final Submission &\\
        \hline
    \end{tabular}
 
\end{center}

\end{document}