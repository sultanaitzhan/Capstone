% Chapter 1

\chapter{Introduction} % Main chapter title

\label{chapter1} % For referencing the chapter elsewhere, use \ref{chapter1}

Often, mathematical modelling of the real-world phenomena results in ordinary and partial differential equations, whose solutions describe the phenomena. Depending on the equations, mathematicians may or may not have the tools to obtain solutions. Fortunately, there are techniques that allow one to analyse differential equations without directly solving them. One such tool is asymptotic analysis, which leads to simplified equations that are similar to original equations. As such, solutions of the simplified equations model the real-world phenomenon, subject to some error estimates.

In particular, one problem that is amenable to asymptotic methods is the \textit{water-wave problem}, which describes the motion of water and its surface under certain conditions. Assuming an irrotational, incompressible, and inviscid fluid, and a domain with flat bottom, the equations of fluid motion are given by
\begin{subequations} \label{S1:DimWholeLineProblem}
\begin{align}
\phi_{xx} + \phi_{zz} &= 0, &-h < z < \eta(x,t), \label{S1:PDE}\\
\phi_{z} &= 0, &z = -h, \label{S1:BBC}\\
\phi_t + g\eta + \frac{1}{2}(\phi_{x}^2 + \phi_{z}^2) &= 0, &z = \eta(x,t), \label{S1:DBC} \\
\eta_t + \phi_{x}\eta_{x} &= \phi_{z}, & z = \eta(x,t), \label{S1:KBC}
\end{align}
\end{subequations}
where $\phi(x,z,t)$ is the unknown velocity potential, $\nabla \phi$ is the fluid velocity, and $\eta(x,t)$ is the unknown surface elevation. In addition, $z$ is the vertical coordinate, $x$ is the horizontal direction, and $g$ is acceleration due to gravity. We let $x\in\RR,$ so that \eqref{S1:DimWholeLineProblem} is the water-wave problem defined \textit{on the whole line}. Although nonlinear partial differential equations (PDEs) \eqref{S1:KBC} and \eqref{S1:DBC} are hard to solve on their own, what makes the problem \eqref{S1:DimWholeLineProblem} truly difficult is the need to solve the Laplace's equation \eqref{S1:PDE} on a domain with an unknown shape. 

To make the equations of motion more tractable, one can reformulate the problem and apply tools of asymptotics. Of particular interest is the work of \cite{AFM2006} (AFM formulation). In this paper, authors rewrite \eqref{S1:DimWholeLineProblem} as a system of two equations, for the surface variable $q(x,t) = \phi(x, \eta(x,t)),$ i.e. the velocity potential evaluated at the surface. Taking advantage of the new formulation, asymptotic reductions are performed in various physical conditions.

One such interesting physical condition is the shallow water regime, which is defined by small-amplitude waves that have a small depth relative to their wavelength. In this regime, asymptotic methods reveal that the fluid motion is governed by the following approximate equations: the \textit{wave} equation,
\begin{equation}\label{S1:eq1}
q_{tt} - q_{xx} = 0,
\end{equation} 
and two \textit{Korteweg de Vries (KdV)} equations,
\begin{equation}\label{S1:eq2}
\begin{aligned}
F_{T} + \frac{1}{3}F_{\xi \xi \xi} + 3 F F_{\xi} &= 0, \\
G_{T} + \frac{1}{3}G_{\zeta \zeta \zeta} + 3 G G_{\zeta} &= 0, \\
\end{aligned}
\end{equation}
where $\xi = x -t, \zeta = x+ t,$ and $q(x,t) = F(\xi, T) + G(\zeta, T).$ Physically, $F$ and $G$ can be interpreted as right-going and left going waves, respectively. The model given by \eqref{S1:eq1} and \eqref{S1:eq2} is called the \textit{KdV model on the whole line}. We observe that on \textit{the half-line}, when waves are bounded from one side by a wall, a rigorous derivation of the corresponding model remains unknown.

In this capstone project, we consider an alternative formulation of \eqref{S1:DimWholeLineProblem}, as presented in \cite{OV2013}. Although slightly different from the AFM formulation, it is contended that this formulation is well-suited for studying asymptotics. We further advocate the efficacy of this formulation by re-deriving the KdV model. 

As a brief outline, in Chapter 2, we introduce the reader to asymptotic and perturbative methods needed for the derivation. In Chapter 3, we explain the physical assumptions of the problem and describe the shallow-water regime. In Chapter 4, we reformulate the problem and derive the approximate equations.  In Chapter 5, we describe an application of the formulation to a water wave problem on \textit{the half line}, while attempting to rigorously derive a half-line model. 
