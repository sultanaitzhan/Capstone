% Preamble
\documentclass[11pt,reqno,oneside,a4paper]{article}
\usepackage[a4paper,includeheadfoot,left=25mm,right=25mm,top=00mm,bottom=20mm,headheight=20mm]{geometry}
\usepackage{siunitx}
\input{texHead}
\author{Sultan Aitzhan}
\title{Final Report Draft}
\newcommand{\theshorttitle}{Final Report Draft}
\date{\today}
\allowdisplaybreaks

\begin{document}
\maketitle
\thispagestyle{fancy}
\tableofcontents
\pagebreak

\section{Introduction}
Often, mathematical modelling of the real-world phenomena results in the ordinary and partial differential equations. Depending on how the equation, mathematicians may or may not have the tools to obtain solutions or understand the phenomenon very well. Fortunately, there exist techniques that allow one to deal with differential equations without solving them. One such tool is asymptotic analysis, which leads to simplified equations that are very similar to actual equations in the limit. As such, solutions of the simplified equations may model the real-world phenomenon, subject to some error estimates.

In particular, one problem that is amenable to asymptotic methods is the \textit{water wave problem}, which describes the behaviour of water and its surface under certain conditions. Assuming an irrotational, incompressible, and inviscid fluid in one dimension, and a domain with flat bottom, the equations of fluid motion are given by
\begin{subequations} \label{S1:DimWholeLineProblem}
\begin{align}
\phi_{xx} + \phi_{zz} &= 0 &-h < z < \eta(x,t) \label{S1:PDE}\\
\phi_{z} &= 0 &z = -h \label{S1:BBC}\\
\eta_t + \phi_{x}\eta_{x} &= \phi_{z} & z = \eta(x,t) \label{S1:KBC}\\
\phi_t + g\eta + \frac{1}{2}(\phi_{x}^2 + \phi_{z}^2) &= 0 &z = \eta(x,t) \label{S1:DBC}
\end{align}
\end{subequations}
where $\phi(x,z)$ is the fluid velocity, $\eta(x)$ is the surface elevation. In addition, $z$ is the vertical coordinate, $x$ is the horizontal direction, and $g$ is acceleration due to gravity. We let $x\in\RR,$ so that \eqref{S1:DimWholeLineProblem} is the water wave problem on the whole line. Although nonlinear partial differential equations (PDEs) \eqref{S1:KBC} and \eqref{S1:DBC} are hard to solve on their own, what makes the problem \eqref{S1:DimWholeLineProblem} truly difficult is the need to solve the Laplace's equation \eqref{S1:PDE} on a domain whose shape is unknown. 

To make the equations of motion more tractable, one can reformulate the problem and apply the tools of asymptotics. Of particular interest is the work \cite{AFM2006}, henceforth referred to as the AFM formulation. In this paper, authors rewrite \eqref{S1:DimWholeLineProblem} as a system of two equations, for the surface variable $q(x) = \phi(x, \eta(x)),$ i.e. the velocity evaluated at the surface. Taking advantage of a new system, various asymptotic reductions are performed. In well-understood physical conditions, the simplified equations correspond to the expected equations.

One interesting physical condition is the shallow water regime, which is defined by small-amplitude waves that have a small depth relative to the wavelength. In this regime, asymptotic tools reveal that the fluid behaviour is governed by the following approximate equations: the \textit{wave} equation
\begin{equation}\label{S1:eq1}
u_{tt} - c_0 u_{xx} = 0,
\end{equation} 
where $c_0$ is the wave velocity, and the \textit{Korteweg de Vries (KdV)} equation
\begin{equation}\label{S2:eq2}
v_{t} + 6(vv_x)_x + v_{xxx} = 0,
\end{equation}
where $u(x,t)$ and $v(x,t)$ are functions related to each other. We refer to the model given by \eqref{S1:eq1} and \eqref{S1:eq2} as the \textit{KdV model}.

In this capstone project, we consider an alternative formulation of the problem \eqref{S1:DimWholeLineProblem}, as presented in \cite{OV2013}. Although slightly different from AFM formulation, it is contended in \cite{OV2013} that this formulation is well-suited for performing asymptotics. We further advocate the efficacy of this formulation, by deriving the equations \eqref{S1:eq1} and \eqref{S1:eq2}. As a brief outline, in Section 2, we introduce the reader to tools of asymptotic analysis, explain the physical assumptions of the problem, and briefly explain why the model \eqref{S1:eq1} and \eqref{S1:eq2} is a good approximation to the water wave problem. In Section 3, we derive the formulation, perform asymptotics, and derive the approximate equations. In Section 4, we justify the utility of the model, contrasting and comparing with other formulations. In Section 5, we describe an application of the formulation to a water wave problem on \textit{the half line}.

\section{Preliminaries}
%In this section, we present various techniques of asymptotics, explain the water wave problem in greater detail, and comment on why we can rely on asymptotics. This chapter assumes familiarity with a basic theory of ordinary and partial differential equations.

\subsection{Water wave problem}
In this section, a sketch of the derivation of the water wave problem will be provided, along with mathematical and physical justification of assumptions made. In addition, we discuss the shallow water limit. Conservation of mass \eqref{S2:CoMass} and conservation of momentum \eqref{S2:CoMomentum} are the two core principles that provide the relevant equations of fluid dynamics
\begin{align}
\frac{\partial \rho}{\partial t} + \nabla \cdot (\rho \V) &= 0, \label{S2:CoMass} \\
\rho\left[ v\frac{\partial \V}{\partial t} + (\V \cdot \nabla) \V\right] &= \textbf{F} - \nabla P + v_{\star} \Delta \V. \label{S2:CoMomentum}
\end{align}
In \eqref{S2:CoMass}-\eqref{S2:CoMomentum}, $\rho = \rho(\X, t)$ denotes the fluid mass density, $\V = \V(\X,t)$ is the fluid velocity, $P$ refers to pressure, $\textbf{F}$ is an external force, and $v_{\star}$ is the kinematic viscosity due to frictional forces. Derivations of \eqref{S2:CoMass} and \eqref{S2:CoMomentum} can be found in \cite[Chapter 3]{Johnson} or \cite[Chapter 1]{CM}. Assuming that the fluid is inviscid, incompressible, and irrotational, one can follow \cite[Section 5.1]{Ablowitz} to obtain the following system:
\begin{subequations} \label{S2:DimWholeLineProblem}
\begin{align}
\phi_{xx} + \phi_{zz} &= 0 &-h < z < \eta(x,t) \label{S2:PDE}\\
\phi_{z} &= 0 &z = -h \label{S2:BBC}\\
\eta_t + \phi_{x}\eta_{x} &= \phi_{z} & z = \eta(x,t) \label{S2:KBC}\\
\phi_t + g\eta + \frac{1}{2}(\phi_{x}^2 + \phi_{z}^2) &= 0 &z = \eta(x,t) \label{S2:DBC}
\end{align}
\end{subequations}
where $\phi(x,z)$ is the fluid velocity, $\eta(x)$ is the surface elevation. In addition, $z$ is the vertical coordinate, $x$ is the horizontal direction, and $g$ is acceleration due to gravity. In deriving the problem, the following assumptions are made:
\begin{itemize}
\item We can write $\V = \nabla \phi.$
\item The problem has 1 horizontal dimension $x \in \RR.$
\item The fluid $|\V|$ tends to equilibrium as $|x| \to \infty.$
\item The surface tension is negligible, i.e. $\sigma = 0.$ 
\item The external force is the buoyancy force, i.e. $\textbf{F} = - \nabla (\rho_0 g z).$
\item The pressure vanishes on the surface, $P = 0$ at $z = \eta(x,t).$
\item The fluid density is constant, i.e. $\rho(\X,t) = \rho_0.$
\end{itemize}
Since the problem is expressed in terms of $\phi,$ and it is a scalar potential of $\V,$ we term the formulation \eqref{S2:DimWholeLineProblem} as the \textit{velocity potential} formulation.

We begin to describe the physical meaning of the problem \eqref{S2:DimWholeLineProblem}. First, we elaborate on each equation:
\begin{itemize}
\item[\eqref{S2:PDE}:] The assumption that the fluid is irrotational means that the curl vanishes:
\[ \nabla \times \V = 0. \]
The conservation of mass then becomes $ \nabla \cdot \V = \nabla \cdot \nabla \phi = 0.$ In other words, \eqref{2PDE} represents that the fluid inside the domain $-h < z < \eta(x,t), x \in \RR$ does not rotate.  
\item[\eqref{S2:DBC}:] This equation is the conservation of momentum applied at the surface $z = \eta(x,t).$ Since conservation of momentum is a statement about the balance of external forces and fluid at the boundary, which is the fluid's surface, this equation describes the dynamics of the velocity potential on the free surface. As such, we term \eqref{S2:DBC} as \textit{the dynamic boundary condition}.
\item[\eqref{S2:KBC}:] This equation represents the condition that the surface $\eta$ is a surface of the fluid. In order words, we require that the surface $\eta$ is always composed of 
fluid particles that remain on the surface. We call \eqref{S2:KBC} \textit{the kinematic boundary condition}, since it is about the geometry and shape of the surface. The condition should be contrasted with the dynamic condition, which is about the interaction of forces acting at the surface. 
\item[\eqref{S2:BBC}:] This equation is an assumption that the bottom is a flat and impermeable surface, so that the fluid cannot escape through the bottom. We call \eqref{S2:BBC} \textit{the bottom condition}. 
\end{itemize}

Second, we explain the physical assumptions and considerations that led to the mathematical formulation of the problem. While mostly following \cite[Chapter 1]{Lannes}, we also expound on some of the considerations and provide additional references. For the most part, the model is the most useful in applications to oceanography.
\begin{itemize}
\item The fluid is assumed to be continuous. We mainly deal with behaviour on scales that are large compared to the distance between molecules that comprise the fluid. Physical quantities such as mass and velocity are thought to be spread continuously throughout the region in question; this is termed as the continuity assumption. This is important to note, since water may have discontinuities in the form of air bubbles. Whenever this happens to a significant degree, e.g. when waves break, the problem \eqref{S2:DimWholeLineProblem} is no longer valid. A microscopic description of fluids is given by the Boltzmann equation. 
\item The fluid is assumed to be incompressible and inviscid. Since the fluid under interest is water, and its density does not change both in space and time \cite{}, water has low compressibility. This justifies homogeneity and incompressibility. In addition, water has low viscosity \cite{}, so we can model the flow with an inviscid model. To account for viscosity, one needs to work with Navier-Stokes equations. 
\item The fluid is assumed to be irrotational. In reality, there are rotational flows, such as tornadoes. In applications, rotational effects are not important, so we do not focus on them \cite{}. The situation is delicate when rotational flows are taken into account; in particular, the assumption $\V = \nabla \phi$ no longer holds, and the velocity formulation should be replaced with the stream-function formulation. See \cite[p.32]{Lannes} for a detailed overview.
\item We assume that the surface and the bottom of the domain can be parametrised as graphs. One may argue that neither the surface nor the bottom need to be (classical) functions; modelling overhanging waves is such example. While this is true, assuming that the surface and the bottom are (classical) functions provides the easiest setting to work in. We can extend the model by considering more general settings \cite{}.
\item The fluid is contained in its domain. We assume that the fluid does not leak through its bottom, nor do the fluid particles leave the fluid surface. These two conditions are given by \eqref{S2:BC} and \eqref{S2:KBC}, respectively.
\item The bottom is assumed to be flat. In reality the bottom may differ drastically. Since the degree of bed rigidity, the porosity, and the roughness all influence the fluid to varying degrees, such bottoms are also more challenging to work with. In addition, when working in the shallow water regime (explained in the next section), it is necessary to make strong assumptions on the bottom to obtain correct asymptotics. As such, we choose to deal with the simplest case of a flat bottom. For a discussion of waves over more realistic seabeds, see \cite[Chapter 9]{DD}.
\item The fluid tends to equilibrium. This is a natural condition so long as one considers infinite domains such as the whole line. It should be noted any discussion of the rate of convergence relates to the \textit{stability} theory of water waves, and will not be touched upon in this project.
\item The water depth is assumed to be bounded by some nonnegative constant. In other words, we always have that $\eta - h \geq 0.$ This is a major limitation, as it excludes vanishes shorelines. However, removing this assumption is an open problem.
\item The external force is conservative. This assumption follows naturally from conservation laws for water waves. Furthermore, for water waves, the most relevant external force are a \textit{body force}, which is the force induced by an outside source and is identical for all fluid particles, and a \textit{local force}, which is the force exerted on a fluid particle by other particles. Gravitational force and pressure is one example of a body force for which one needs to account. The local force is due to friction. Since the fluid is inviscid, no local force is present. See \cite[Chapter 4]{CK} for a detailed discussion.
\item The surface tension is negligible and the surface pressure is constant. In applications to coastal oceanography, the surface tension tends to be very small, which justifies the assumption. Of course, surface tension is relevant when describing some smaller scale phenomena such as ripples. Since we expect no large weather variations, we can assume that locally pressure is constant. If one is to incorporate non-constant pressure and/or surface tension, the dynamic condition \eqref{S2:DBC} needs to be changed accordingly.
\item The system is two dimensional, that is, there is one spatial dimension $x$ and one vertical dimension $z.$ By ignoring the remaining dimension $y,$ we are making an assumption that the fluid is only moving in $x$ direction. In other words, we consider the special case of \textit{no transverse waves}.  Incorporating weak transverse variation, leads to a generalisation of the KdV equation, the Kadomtsev-Petviashvili equation, equation (5.33) in \cite[Chapter 5]{Ablowitz}:
\begin{equation}\label{S2:KP}
\partial_x(u_t + 6uu_x + u_{xxx}) + 3 \rho u_{yy} = 0.
\end{equation}
In \eqref{S2:KP}, if $\rho = 1,$ then water waves with small surface tension are modelled, and if $\rho = -1,$ then water waves with large surface tension are modelled.
\end{itemize}
Finally, we discuss the role of initial conditions in the water waves problem. While the wave motion is expected to be initiated in some fashion, we are mainly interested in the evolution of the wave motion in many and varied situations. As such, initial conditions are not mentioned explicitly. 

\subsection{Asymptotic analysis and perturbation methods}
In this section, we present an introduction to perturbation theory, and its most relevant technique, namely the multiple scale analysis. The focus is on the illustration of ideas through examples, rather than rigorous justification and proofs. Examples used are from Chapters 7 and 11 of \cite{BO}.

Perturbation theory is a collection of iterative techniques used for obtaining approximate solutions to problems involving a small parameter $\epsilon$. The main idea is to represent the unknown variable as a \textit{perturbation series}, which is a formal power series 
\begin{equation}\label{S2:pertubationseries}
f = f_0 + \epsilon f_1 + \epsilon^2 f_2 + \ldots.
\end{equation} 
Substituting \eqref{S2:pertubationseries} into the original problem decomposes what is a difficult problem  into many easier ones. Solving for the first $n$ terms in the series, one obtains the approximated solution 
\[f \approx f_0 + \epsilon f_1 + \ldots + \epsilon^n f_n.\]
It should be recognised that with this approach, perturbation theory is most useful when the first few problems reveal the important features of the solution and the remaining problems give only small corrections. The perturbative techniques are applied in numerous settings, including finding roots of polynomials and solving initial value problems for differential equations.

Since we deal with the differential equations, we illustrate the method with an ODE.

\eg{
Consider the following initial-value problem:
\[ y'' + y = \frac{\cos(x)}{3 + y^2}, \qquad y(0) = y\left( \frac{\pi}{2} \right) = 2. \]
}\label{S2:example}

Observe that the perturbation series obtained in Example \ref{S2:example} converges uniformly as $\epsilon \to 0.$ However, in general, one need not expect uniform convergence. A paradigmatic example is the weakly nonlinear Duffing oscillator
\begin{equation}\label{Duffing}
\frac{d^2 y}{d t^2} + y + \epsilon y^3 = 0, \qquad y(0)=1 \qquad y'(0)=0.
\end{equation}
Application of the regular perturbation series yields
\begin{equation*}
y(t) \approx y_0 + \epsilon y_1 = \cos t + \epsilon\left(\frac{1}{32} \cos 3t -\frac{1}{32} \cos t + \frac{3}{8} t \sin t\right ),
\end{equation*}
which converges as $\epsilon \to 0$ for fixed $x.$ However, one must observe that the convergence is point-wise, but not uniform. Indeed, for values $t \sim 1/\epsilon$ or larger, the presence of the $t \sin t$ term in $y_1$ implies that the amplitude of oscillation grows in $t;$ we call such terms \textit{secular}. However, solutions of the Duffing oscillator are known to be bounded. In particular, this suggests that the usual perturbation expansion of $y$ is not sufficient, and that the secularity is an outcome of this misfortune.

\subsubsection*{Multiple scale analysis}
When ordinary perturbative methods fail to give a uniformly accurate approximation, the method of \textit{multiple scales} comes in handy. The idea behind the multiple scales is to introduce a new variable $\tau = \epsilon t.$ Physically, $\tau$ represents a longer time scale than $t$, since $\tau$ is not negligible when $t$ is of order $1/\epsilon$ or larger. Observe that although $y(t)$ is a function of $t$ alone, through introduction of $\tau,$ $y(t)$ becomes a function of $t$ and $\tau,$ i.e $y(t, \tau).$ As such, the multiple scales looks for solutions as functions of both $t$ and $\tau,$ treating these variables independently. It must be emphasised that such treatment in two variables is rather an artifice, needed to eliminate secularities.

We illustrate the method on the Duffing oscillator. Formally, we write
\[ y(t) = Y_0(t, \tau) + \epsilon Y_1(t, \tau) + \mathcal{O}(\epsilon^2) \]
By chain rule, we have
\[ \frac{\D}{\D t} = \frac{\partial}{\partial t} + \epsilon \frac{\partial}{\partial \tau}. \]
Note
\begin{align}
\frac{\D y}{\D t} &= \frac{\partial Y_0}{\partial t} + \epsilon \left( \frac{\partial Y_0}{\partial \tau} + \frac{\partial Y_1}{\partial t} \right) + \mathcal{O}(\epsilon^2), \label{1stD} \\
\frac{\D^2 y}{\D t^2} &=\frac{\partial^2 Y_0}{\partial t^2} + \epsilon \left( 2 \frac{\partial^2 Y_0}{\partial \tau \partial t} + \frac{\partial^2 Y_1}{\partial t^2} \right) + \mathcal{O}(\epsilon^2). \label{2ndD}
\end{align}
Substituting \eqref{1stD} and \eqref{2ndD} into \eqref{Duffing} and collecting in powers of $\epsilon$ yields
\begin{align}
\frac{\partial^2 Y_0}{\partial t^2} + Y_0 &= 0, \label{S1}\\
\frac{\partial^2 Y_1}{\partial t^2} + Y_1 &= - Y_0^3 - 2 \frac{\partial^2 Y_0}{\partial \tau \partial t}. \label{S2}
\end{align}
The general solution of \eqref{S1} is 
\begin{equation}\label{S3}
Y_0(t, \tau) = A(\tau) e^{it} + A^*(\tau) e^{-it},
\end{equation}
where $A(\tau)$ is an arbitrary complex function in $\tau.$ We determine $A(\tau)$ by requiring that secular terms do not appear in $Y_1(t, \tau).$ Substitution of \eqref{S3} into \eqref{S2} gives
\begin{equation}\label{S4}
\frac{\partial^2 Y_1}{\partial t^2} + Y_1 =  \left(-3 A^2 A^* - 2i \frac{\D A}{\D \tau}\right) e^{it} + \left(-3 A (A^*)^2 + 2i \frac{\D A^*}{\D \tau}\right) e^{-it}  - A^3 e^{3it}  - (A^*)^3 e^{-3it}.
\end{equation}
We have already seen that $e^{it}$ and $e^{-it}$ appear in the solution of \eqref{S1}, which is the homogeneous version of \eqref{S2}. Therefore, unless the coefficients of $e^{it}$ and $e^{-it}$ are zero, the solution $Y_1$ will be secular in $\tau.$  To preclude secularity, we need $A$ needs to satisfy
\begin{align*}
-3 A^2 A^* - 2i \frac{\D A}{\D \tau} &= 0, \\
-3 A (A^*)^2 + 2i \frac{\D A^*}{\D \tau} &= 0.
\end{align*}
Observe that the two equations are complex conjugate of each other, so $A(\tau)$ is not overdetermined. Solving for $A(\tau)$ along with initial conditions $y(0)=1, y'(0)=0$ yields 
\[ A(\tau) = \frac{1}{2} e^{i3\tau/8}.\]
Thus, T\eqref{S3} becomes 
\[ Y_0(t, \tau) = \cos (t + \frac{3}{8}\tau).\]
Finally, using that $\tau = \epsilon t,$ we obtain the approximated solution 
\begin{equation*}
y(t) = \cos \left[ t(1 + \epsilon\frac{3}{8})\right] + \mathcal{O}(\epsilon), \qquad \epsilon \to 0, \quad \epsilon t = \mathcal{O}(1).
\end{equation*}
To conclude, we remark that the choice of scales $\tau = \epsilon t$ tends to be example-specific. More generally, one may choose $\tau = f(t),$ where $f$ can be any function. For example, in 
\[ y''(t) + y - \epsilon t y= 0, \qquad y(0) = 1, \qquad y'(0) = 0\]
one would use $\tau = \sqrt{\epsilon} t,$ and for
\[ y''(t)+ \omega^2(\epsilon t) y= 0,\]
one would use $\tau = \int^t \omega (\epsilon s) \D s.$

\rmk{It is important to understand the need for multiple scales. In the example of the Duffing oscillator, we could solve the problem numerically for the exact solution, or approximate the solution via multiple scales. A question arises: why use multiple scales when we can solve the differential equations numerically? 

Note that most of the real-world phenomena are expressed in terms of PDEs, which are much harder to solve numerically than ODEs. Due to many differences in the physical phenomena, there is no unified analytic and numerical treatment. In addition, developing numerical schemes can be very tricky, since issues such as stability, error as well as the physical conditions need to be carefully addressed to obtain an effective software. Furthermore, the cost of numerically solving PDEs can be expensive, in terms of time required to find a solution and computational power needed if the great precision is required. The latter is particularly important for real time predicting. This is another reason to prefer multiple scales. As mentioned before, multiple scales and perturbation methods, when appropriately applied, turn the original, difficult problem into many easier problems. These problems are much easier to solve, and provide further insight into the physics of the problem.}

\subsection{Are asymptotic and perturbative methods reliable?}
Given the emphasis on asymptotic and perturbative methods, one is interested whether these methods provide ``reliable" solutions. Formally, how can we justify that solutions obtained from asymptotic equations converge to the solutions of the original problem? In the context of the water waves problem, we wonder if the KdV model, provided by the shallow water approximation, is ``reasonable". Of course, in asking these questions, one needs to specify the meaning of ``reliable" and ``reasonable". Following \cite{Lannes}, the validity of our asymptotic model can be understood from the following questions:
\begin{enumerate}
\item Do the solutions of the water waves problem exist on the required time scale?
\item Do the solutions of the asymptotic, KdV model exist on the same time scale?
\item Are the asymptotic solutions close to the actual solutions with the corresponding initial data? If so, how close?
\end{enumerate}
If the answer to all three questions is positive, then the asymptotic model is \textit{fully justified}. Indeed, the KdV model is fully justified (see \cite[p. 297-298]{Lannes}). However, actual proofs of the answers are involved and require advanced mathematics (see Chapter 7 and Appendix C in \cite{Lannes}). Since the mathematical justification of the KdV model is beyond the scope of the capstone project, we choose not to discuss this topic.

\subsection{Shallow water regime} 
For now, the problem \eqref{S2:DimWholeLineProblem} admits numerous types of water waves: short waves, long waves, intermediate waves. In particular, we have yet to specify how water wavelength relates to the water depth. As such, we first examine the \textit{dispersion relation}, to understand the relation between wave velocity and wavelength. With this in mind, we focus on the shallow water regime, characterised by small-amplitude waves that have long wavelength, relative to the water depth. In nature, tsunamis and tidal waves are examples of this regime. In coastal engineering, this regime has implications in the design of harbours and in studying estuaries and lagoons. 

First, we consider small amplitude waves, or equivalently, we assume that $|\eta| \ll 1$ and $\Vert \nabla \phi \Vert \ll 1.$ Dispersion relation is obtained by linearisation of the problem \eqref{S2:DimWholeLineProblem} around $z=0.$ More concretely, we begin by assuming the special form of the solutions
\[
\phi_s(x,z,t) = A(k,z,t) \exp(ikx) 
\]
and 
\[ 
\eta_s(x,t) = \tilde{\eta}(k,t)\exp(ikx).
\]
We then may follow Section 5.2 of \cite{Ablowitz}, to obtain the following ODE:
\[ 
\frac{\partial^2 \tilde{\eta}}{\partial t^2} + g k \tanh(k h) \tilde{\eta} = 0.
\]
Assuming that $ \tilde{\eta}(k,t) = \tilde{\eta}(k, 0) \exp(-i \omega t)$ yields the dispersion relation
\[ 
\omega^2 = g k \tanh(k h).
\]
Here, $\omega$ is a wavelength, $k$ is a wave number, and $g$ is gravity. For shallow water, the wavelength $\omega$ is much bigger than the depth $h,$ so $k$ is very small. Therefore, $kh \ll 1,$ and expansion of $\tanh(kh)$ in $kh$ leads to
\[ \omega^2 = gk(kh - \frac{(kh)^3}{3} + \ldots)  \simeq ghk^2. \]
Thus, small amplitude water waves in shallow water have wavelength $ \omega = \pm \sqrt{gh}|k|,$ or equivalently, velocity $c_0 = \sqrt{gh}.$ 

The dispersion relation allows to determine the properties of water waves that we would like to model. In particular, knowing the velocity $c_0 = \sqrt{gh}$ allows us to \textit{rescale} the problem \eqref{S2:DimWholeLineProblem}, so that the rescaled problem models shallow water waves.

In addition to admitting numerous types of water waves, the problem \eqref{S2:DimWholeLineProblem} does not specify the dimensions of the problem. Since dimensions of the problem are directly related to the units of variables (wavelength, time, height), it can be difficult to decide which terms are negligible when performing an approximation procedure. The process of \textit{non-dimensionalisation} removes the dimensions of the problem, allowing us to work with ``pure" numbers. We define dimensionless variables as follows:
\begin{equation}\label{S2:NDvar}
z = hz' \qquad x = \lambda x' \qquad t = \frac{\lambda}{c_0}t' \qquad \eta = a \eta' \qquad \phi = \frac{\lambda g a}{c_0} \phi',
\end{equation}
where $c_0 = \sqrt{gh}$ is the shallow water speed, $\lambda$ is the typical wavelength of the initial data, and $a$ is the maximum of typical amplitude of initial data. See \cite[Sections 1.3.2-1.3.3]{Lannes} for a detailed discussion of \eqref{S2:NDvar}. We further define the following parameters 
\[ \epsilon = \frac{a}{h}, \qquad \mu = \frac{h}{\lambda}.\]
Physically, $\epsilon$ is an amplitude of the water wave, while $\mu$ is a ratio of depth to a typical wavelength. Alternatively, we regard that $\epsilon$ measures nonlinearity and $\mu$ measures dispersion. Transforming the problem \eqref{S2:DimWholeLineProblem} via chain rule and dropping the primed notation yields
\begin{subequations}\label{S2:WLPND1}
\begin{align}
\label{S2:PDEND1}  \mu^2 \phi_{xx} + \phi_{zz} &= 0 &-1 <&z < \varepsilon\eta \\
\label{S2:BC1ND1} \phi_z &= 0 &z &= -1  \\ 
\label{S2:BC2ND1} \phi_{t} + \frac{\varepsilon}{2} \left(\phi_{x}^2 + \frac{1}{\mu^2}\phi_{z}^2\right) + \eta &= 0 &z &= \varepsilon\eta(x,t)\\
\label{S2:BC3ND1} \mu^2 \left[\eta_{t} + \varepsilon \phi_{x} \eta_{x}\right] &= \phi_{z} &z &= \varepsilon\eta(x,t).
\end{align}
\end{subequations}
The problem \eqref{S2:WLPND1} is a ``normalised" problem that models shallow water waves.

Observe that we have yet to make any assumptions about the parameters $\epsilon$ and $\mu,$ nor have we prescribed any relationship between the two parameters. To obtain interesting limiting equations, we make the following assumptions:
\begin{itemize}
\item Assume $\mu \ll 1.$ Recall that $\mu$ is a ratio of depth to wavelength, and in shallow water regime, we expect depth is much smaller compared to wavelength. This justifies the assumption.
\item To obtain equations that are interesting, we should balance the parameters by connecting them to each other. This is known as the Kruskal's principle of maximal balance. We elect to choose $\varepsilon = \mu^2,$ which reflects the balance of weak nonlinearity and weak dispersion.
\item From the maximal balance, it follows that $\epsilon \ll 1.$ Physically, water waves have small amplitude, and this is the first assumption used in deriving the dispersion relation.
\end{itemize}
Thus, the nondimensional problem \eqref{S2:WLPND1} becomes:
\begin{subequations}\label{S2:WLPND2}
\begin{align}
\label{S2:PDEND2}  \varepsilon\phi_{xx} + \phi_{zz} &= 0 &-1 <&z < \varepsilon\eta \\
\label{S2:BC1ND2} \phi_z &= 0 &z &= -1  \\ 
\label{S2:BC2ND2} \phi_{t} + \frac{1}{2} \left(\varepsilon\phi_{x}^2 + \phi_{z}^2\right) + \eta &= 0 &z &= \varepsilon\eta(x,t)\\
\label{S2:BC3ND2} \varepsilon\left[\eta_{t} + \varepsilon \phi_{x} \eta_{x}\right] &= \phi_{z} &z &= \varepsilon\eta(x,t).
\end{align}
\end{subequations}
Note that there is no reason not to balance in other ways, say $\varepsilon = \sqrt{\mu}.$ There are many options, and some of them will give interesting equations, while others do not lead to anything interesting. It is this assumption in the procedure that determines the relevance of to-be-derived equations.

Finally, we describe the chief result that we seek to obtain in this project. Let us assume an expansion
\[ \phi(x,z,t) = \phi_0(x,z,t) + \epsilon \phi_1(x,z,t) + \mathcal{O}(\epsilon^2).\]
Substitution of the perturbation series into \eqref{S2:PDEND2} and \eqref{S2:BC1ND2} yields that an approximation
\begin{equation}\label{S2:PS0}
\phi = A - \frac{\epsilon}{2} A_{xx}(z+1)^2 + \frac{\epsilon^2}{4!} A_{xxxx} (z+1)^4 + \mathcal{O}(\epsilon^3),
\end{equation}
where
\[ \phi_0 = A(x,t).\]
This approximation is valid in $-1<z<\epsilon \eta.$ Substitution of the series \eqref{S2:PS0} into \eqref{S2:BC2ND2} and \eqref{S2:BC3ND2}, along with appropriate manipulations yields
\begin{equation}\label{S2:PS1}
A_{tt} - A_{xx} = \epsilon\left( \frac{A_{xxxx}}{3} - 2A_x A_{xt} - A_{xx}A_t\right),
\end{equation}
valid up to $\mathcal{O}(\epsilon).$ Assume an expansion for $A:$ 
\[ A = A_0 + \epsilon A_1 + \mathcal{O}(\epsilon^2);\]
substituting this expansion into \eqref{S2:PS1} yields
\begin{equation}\label{S2:W1}
 A_{0tt} - A_{0xx} = 0.
\end{equation}
This is the wave equation, valid within $\mathcal{O}(\epsilon^0).$ The general solution is $A_0 = F(x-t) + G(x+t),$ for some general functions $F,G.$

We would like to determine the functions $F,G.$ We first observe that \eqref{S2:PS1} has secular terms: this can be shown directly via dispersion relation, or one could solve \eqref{S2:PS1} numerically and see that the solution is unbounded in time. The presence of secular terms warrants an introduction of time scales:
\[ \tau_0 = t, \qquad \tau_1 = \epsilon t.\]
We also write 
\[ 
\xi = x- \tau_0, \qquad \zeta = x + \tau_0,
\]
so that $A_0 = A_0(\xi, \zeta, \tau_1) = F(\xi, \tau_1) + G(\zeta, \tau_1).$ Via appropriate calculations, within $\mathcal{O}(\epsilon)$, we must have 
\begin{align}
2F_{\tau_1} + \frac{1}{3}F_{\xi\xi\xi} + 3 F F_\xi &= 0 \label{S2:KdV1} \\
2G_{\tau_1} - \frac{1}{3}G_{\zeta\zeta\zeta} -  3 G G_\zeta &= 0. \label{S2:KdV2}
\end{align}
In other words, we have obtained two KdV equations, \eqref{S2:KdV1} and \eqref{S2:KdV2}, for $F$ and $G,$ which determine the leading order solution $A_0$ of the problem. 

The wave and KdV equations are special PDEs. The wave equation arises as a model in numerous fields of classical physics, such as electrodynamics, plasma physics, and general relativity. The KdV equation appears whenever long waves propagate over dispersive medium, be it fluid mechanics, nonlinear optics, or Bose-Einstein condensation. Because of how they occur independently of the applications, wave and KdV have been studied extensively. Furthermore, KdV is special: despite being nonlinear, the PDE can be solved exactly, by means of inverse scattering transform. 

In conclusion, the leading order solution of the water wave problem for small-amplitude, shallow water waves is described by the wave equation \eqref{S2:W1} and two KdV equations \eqref{S2:KdV1}, \eqref{S2:KdV2}. This is the result we seek to obtain. 


\section{Non-local derivation on the whole line}
Recall that the equations of fluid motion are challenging to work with directly, due to the nonlinear boundary conditions and the unknown domain. In addition, these complications lead to difficulties when attempting to deal with questions of existence and well-posedness. To address these issues, reformulations of the problem have been introduced: they result in equivalent problems that are more tractable. While such reformulations can be helpful, they may suffer from other issues. Below, we give a short overview of these formulations, along with explaining the pros and cons of each. Our main goal is to look for a reformulation in the water surface $\eta$, and that generalises easily to two-dimensional problem. We chose this criteria with a view towards applications: indeed, in applications, determining the water surface $\eta$ is the main interest.

For example, for one-dimensional surfaces (no $y$ variable), conformal mappings can be used to eliminate these problems (for an overview, see \cite{DKSZ}). However, this approach is limited to one-dimensional surfaces. For both one- and two-dimensional surfaces, other formulations (such as the Hamiltonian formulation given in  \cite{Zakharov} or the Zakharov–Craig–Sulem formulation, \cite{CS1993}) reduce the Euler equations to a system of two equations, in terms of surface variables $q = \phi(x, \eta)$ and $\eta$ only, by introducing a Dirichlet-to-Neumann operator (DNO). However, using their formulation one must truncate the series expansion of the DNO.

A new non-local formulation is introduced in \cite{AFM2006}, (henceforth referred to as the AFM formulation) that results in a system of two equations for the same variables as in the DNO formulation. Both the DNO and AFM formulations reduce the problem from the full fluid domain to a system of equations that depend on the surface elevation $\eta(x)$ and the velocity potential evaluated at the surface, $q(x) = \phi(x, \eta).$ While this simplification reduces the computational domain, the equations require solving for an additional function $q,$ which is typically of less interest and not easily measured in experiments. 

A new formulation is introduced in \cite{OV2013}, which reduces the water waves problem to a system of two equations, in one variable $\eta.$ This formulation allows to rigorously investigate one- and two-dimensional water waves. The computation of Stokes-wave asymptotic expansions for periodic waves justifies the use of the formulation; indeed, following \cite{OV2013}, the computations can be performed with arguably less effort, especially for two-dimensional waves. Our goal is to further justify the use of this formulation, which we call the $\mathcal{H}$ formulation.

In this section, we first rewrite the water wave problem by introducing a normal-to-tangential operator. We then perform a perturbation expansion for the operator, and proceed to obtain an expression for the surface elevation. Finally, performing asymptotics and applying time scales yields the desired approximate equations. We emphasise that it is not our intention here to further the study of the water wave problem per se, but rather to demonstrate the efficacy of the $\mathcal{H}$ formulation for doing asymptotics. 

\subsection{Water-wave problem on the whole line: non-local formulation}

Recall the water wave problem 
\begin{subequations} \label{S3:DimWholeLineProblem}
\begin{align}
\phi_{xx} + \phi_{zz} &= 0 &-h < z < \eta(x,t) \label{S3:PDE}\\
\phi_{z} &= 0 &z = -h \label{S3:BBC}\\
\eta_t + \phi_{x}\eta_{x} &= \phi_{z} & z = \eta(x,t) \label{S3:KBC}\\
\phi_t + g\eta + \frac{1}{2}(\phi_{x}^2 + \phi_{z}^2) &= 0 &z = \eta(x,t) \label{S3:DBC}
\end{align}
\end{subequations}
and consider the velocity potential evaluated at the surface:
\[ 
q(x,t ) = \phi (x, \eta(x, t)).
\]
We seek to reformulate the problem \eqref{S3:DimWholeLineProblem}. Combining \eqref{S3:KBC} and \eqref{S3:DBC}, evaluated at $z = \eta$, we obtain 
\begin{equation}\label{S3:eq1}
q_t + \frac{1}{2}q_x^2 + g \eta - \frac{1}{2} \frac{(\eta_t + q_x \eta_x)^2}{1 + \eta_x^2} = 0,
\end{equation}
which is an equation for two unknowns $q ,\eta.$ We need an equation in one unknown only. 

Given the domain $D = \RR \times (-h, \eta)$, let 
\[  
\vec{N} = \begin{bmatrix} -\eta_x \\ 1 \end{bmatrix} \qquad
\mbox{and} \qquad
\vec{T} =\begin{bmatrix} 1 \\ \eta_x \end{bmatrix} \]
be vectors normal and tangent to the surface $D,$ respectively. We introduce an operator that maps the normal derivative at a surface $\eta$ to the tangential derivative at the surface:
\begin{equation}\label{S3:defH1}
\mathcal{H}(\eta, D) \{ \nabla \phi \cdot \vec{N} \} = \nabla \phi \cdot \vec{T},
\end{equation}
where $D = - i \nabla.$ For convenience, we drop the vector notation. Note that by \eqref{S3:KBC}, 
\[ 
\nabla \phi \cdot N = \begin{bmatrix} \phi_x \\ \phi_z \end{bmatrix} \cdot \begin{bmatrix} -\eta_x \\ 1 \end{bmatrix} = \phi_z - \phi_x \eta_x = \eta_t,
\] 
and by chain rule,
\[ 
\nabla \phi \cdot T = \begin{bmatrix} \phi_x \\ \phi_z \end{bmatrix} \cdot \begin{bmatrix} 1 \\ \eta_x \end{bmatrix} = \phi_x + \eta_x \phi_z = q_x.
\]
This allows us to rewrite \eqref{S3:defH1} as 
\begin{equation}\label{S3:defH2}
\mathcal{H}(\eta, D) \{ \eta_t \} = q_x,
\end{equation}
and obtain a system
\begin{align*}
q_t + \frac{1}{2}q_x^2 + g \eta - \frac{1}{2} \frac{(\eta_t + q_x \eta_x)^2}{1 + \eta_x^2} &= 0, \\
\mathcal{H}(\eta, D) \{ \eta_t \} &= q_x.
\end{align*}
Differentiate \eqref{S3:eq1} with respect to $x$ and \eqref{S3:defH2} with respect to $t:$
\begin{align}
\partial_t(q_x) + \partial_x\left(\frac{1}{2}q_x^2 + g \eta - \frac{1}{2} \frac{(\eta_t + q_x \eta_x)^2}{1 + \eta_x^2}\right) &= 0, \label{S3:eq2} \\
\partial_t(\mathcal{H}(\eta, D) \{ \eta_t \}) &= q_{xt}. \label{S3:defH3}
\end{align}
Substituting \eqref{S3:defH3} into \eqref{S3:eq2}, we obtain 
\begin{equation}\label{S3:Hequation}
\partial_t\left(\mathcal{H}(\eta, D)\{ \eta_t\} \right) + \partial_x\left( \frac{1}{2}\left(\mathcal{H}(\eta, D)\{\eta_t\} \right)^2 + \epsilon \eta - \frac{1}{2} \frac{(\eta_t + \eta_x \mathcal{H}(\eta, D)\{ \eta_t\})^2}{1+\eta_x^2}\right) = 0.
\end{equation}
Equation \eqref{S3:Hequation} represents a scalar equation for the water wave surface $\eta.$ The utility of \eqref{S3:Hequation} depends on whether we can find a useful representation for the operator $\mathcal{H}(\eta, D).$ In the next section, we proceed to find an equation that the $\mathcal{H}$ operator must satisfy. 

\subsubsection{Behaviour of the $\mathcal{H}$ operator: the whole line.}
Consider the following boundary value problem:
\begin{subequations}
\begin{align}
\phi_{xx} + \phi_{zz} &= 0 &-h < z < \eta(x,t) \label{S3:PDE1}\\
\phi_{z} &= 0 &z = -h \label{S3:BBC}\\
\nabla  \phi \cdot N &= f(x )& z = \eta(x,t) \label{S3:KBC}
\end{align}
\end{subequations}
Let $\varphi$ be harmonic on $D.$ Using \eqref{PDEn} and that $\varphi_z$ is also harmonic on $D,$ we have
\[ \varphi_z(\phi_{xx} + \phi_{zz}) - \phi((\varphi_z)_{zz} + (\varphi_{z})_{xx}) = 0. \]
Taking the integral over the domain yields
\[ \int^{\infty}_{-\infty} \int^{\eta(x)}_{-h} \varphi_z(\phi_{xx} + \phi_{zz}) - \phi((\varphi_z)_{zz} + (\varphi_{z})_{xx}) \D z \D x = 0.\]
An application of Green's theorem gives 
\begin{equation}\label{S3:dInt}
\int_{\partial D} \varphi_z(\nabla  \phi \cdot \textbf{n}) - \phi(\nabla  \varphi_z \cdot \textbf{n}) \D s = 0,
\end{equation} 
where $\partial D$ is the boundary of the domain, $\D s$ is the area element, and $\textbf{n}$ is the normal vector. Now, observe that
\[- \nabla \varphi_z \cdot \textbf{n} = \nabla  \varphi_x \cdot \textbf{t}, \]
%\begin{align*} - \nabla  \varphi_z \cdot N = - \begin{pmatrix} \varphi_{zx} \\ \varphi_{zz} \end{pmatrix} \cdot \begin{pmatrix} - \dfrac{\D z}{\D s} \\ \dfrac{\D x}{\D s} \end{pmatrix} &=  - \begin{pmatrix} \varphi_{zx} \\ - \varphi_{xx} \end{pmatrix} \cdot \begin{pmatrix} - \dfrac{\D z}{\D s} \\ \dfrac{\D x}{\D s} \end{pmatrix} \\ &= \begin{pmatrix} \varphi_{zx} \\ \varphi_{xx} \end{pmatrix} \cdot \begin{pmatrix} \dfrac{\D z}{\D s} \\ \dfrac{\D x}{\D s} \end{pmatrix} \\ &= \begin{pmatrix} \varphi_{xx} \\ \varphi_{xz} \end{pmatrix} \cdot \begin{pmatrix} \dfrac{\D x}{\D s} \\ \dfrac{\D z}{\D s} \end{pmatrix} \\&= \nabla  \varphi_x \cdot \textbf{t}, \end{align*}
where $\textbf{t}$ is the tangential vector. We use this to rewrite \eqref{S3:dInt} and obtain the following contour integral:
\begin{equation}
\begin{aligned}
0 &= \int_{\partial D} \varphi_z(\nabla  \phi \cdot N) + \phi(\nabla  \varphi_z \cdot \textbf{t}) \D s \nonumber \\
&= \int_{\partial D} \varphi_z (\phi_z \D x - \phi_x \D z) + \phi(\varphi_{xx} \D x + \varphi_{xz} \D z) \label{S3:DimContInt}
\end{aligned}
\end{equation}
We split the contour into four segments:
\begin{align*}
\int_{\partial D} &= \int^{\infty}_{-\infty} \bigg|^{z = -h} + \int_{-h}^{\eta(x)} \bigg|^{x \to \infty}+ \int_{\infty}^{-\infty}\bigg|^{z=\eta(x)} + \int_{\eta(x)}^{-h}\bigg|^{x \to -\infty} \\
&=  \int^{\infty}_{-\infty} \bigg|^{z = -h} +  \int_{-h}^{\eta(x)} \bigg|^{x \to \infty} - \int_{-\infty}^{\infty}\bigg|^{z=\eta(x)} -  \int^{\eta(x)}_{-h} \bigg|^{x \to -\infty}.
\end{align*}
Consider each segment:
\begin{itemize}
\item As $|x|\to \infty,$ we know that $\phi$ and its gradient vanish, so the integrals
\[  \int_{-h}^{\eta(x)} \bigg|^{x \to \infty}, \int^{\eta(x)}_{-h} \bigg|^{x \to -\infty}\]
vanish.
\item At $z = -h, \D z = 0,$ so we have 
\begin{align*}
\int^{\infty}_{-\infty}\varphi_z (\phi_z \D x - \phi_x \D z) &+ \phi(\varphi_{xx} \D x + \varphi_{xz} \D z) \\
&= \int^{\infty}_{-\infty}\varphi_z \phi_z + \phi \varphi_{xx} \D x \\
&= \int^{\infty}_{-\infty} \phi \varphi_{xx} \D x \qquad \text{(since $\phi_z=0$ at $z =-h$)} \\
&= \phi(x,-h) \varphi_x(x, -h) \bigg|^{\infty}_{-\infty} - \int^{\infty}_{-\infty} \phi_x(x,-h)\varphi_x(x,-h) \D x \\
&= 0,
\end{align*}
where we pick $\varphi$ such that $\varphi_x(x, -h) = 0.$
\item At $z = \eta, \D z = \eta_x \D x,$ so we have 
\begin{align*}
\int_{-\infty}^{\infty} \varphi_z(\phi_z - \phi_x \eta_x ) &+ \phi(\varphi_{xx}  +  \varphi_{xz} \eta_x) \D x \\
&= \int_{-\infty}^{\infty} \varphi_z(\begin{pmatrix} \phi_x  \\ \phi_z \end{pmatrix}\cdot \begin{pmatrix} - \eta_x \\ 1 \end{pmatrix} + \phi \frac{\D \varphi_x(x, \eta)}{\D x} \D x \\
&= \int_{-\infty}^{\infty} \varphi_z \nabla \phi \cdot N+ \phi \dfrac{\D \varphi_x(x, \eta)}{\D x} \D x \\
&= \int_{-\infty}^{\infty} \varphi_z \nabla \phi \cdot N -\varphi_x \dfrac{\D \phi(x, \eta)}{\D x} \D x \qquad \text{(integration by parts)}\\
&= \int_{-\infty}^{\infty} \varphi_z \nabla \phi \cdot N -\varphi_x \begin{pmatrix} \phi_x  \\ \phi_z \end{pmatrix} \cdot \begin{pmatrix} 1 \\ \phi_x \eta_x \end{pmatrix} \D x \\
&= \int_{-\infty}^{\infty} \varphi_z \nabla \phi \cdot N -\varphi_x \nabla \phi \cdot T \D x \\
&= \int_{-\infty}^{\infty} \varphi_z f(x) -\varphi_x(x, \eta) \mathcal{H}( \eta, D)\{ f(x) \} \D x \qquad \text{(recall \eqref{KBCn} and \eqref{defH1})}. 
\end{align*}
\end{itemize}
Combining segments, we obtain: 
\begin{equation}\label{S3:dInt2}
\int_{-\infty}^{\infty} \varphi_z f(x) -\varphi_x(x, \eta) \mathcal{H}( \eta, D)\{ f(x) \} \D x = 0.
\end{equation}
Note that $\varphi(x,z) = e^{-ikx} \sinh(k(z+h)), k \in \RR$ is one solution of 
\[ \Delta \varphi = 0, \qquad \varphi_z(-h,z) = 0.\]
Then, \eqref{S3:dInt2} becomes
\begin{align*}
\int_{-\infty}^{\infty} e^{-ikx}( k \cosh(k(\eta+h)) f(x) + ik \sinh(k(\eta+h)) \mathcal{H}( \eta, D)\{ f(x) \}) \D x = 0.
\end{align*}
It can be shown that we can take out $k$ in the integral, so that the below holds for all $k \in \RR.$
\begin{equation}\label{S3:DimHbehav}
\int_{-\infty}^{\infty} e^{-ikx}( i  \cosh(k(\eta+h)) f(x) - \sinh(k(\eta+h)) \mathcal{H}( \eta, D)\{ f(x) \}) \D x = 0, \qquad k \in \RR.
\end{equation}
The equation \eqref{S3:DimHbehav} gives a description of how the operator $\mathcal{H}( \eta, D)$ behaves, in dimensional coordinates. 
%\rmk{Even though \eqref{DimHbehav} and \eqref{NDimHbehav} hold for all $k \in \RR,$ the case $k = 0$ still poses some challenges. Namely, the first term of $\mathcal{H}$ will contain $\coth (uk)$ term, which blows up as $k \to 0.$ As will be seen, this problem should be dealt by selecting an appropriate function space for $\eta.$}

In summary, introduction of the normal-to-tangential operator $\mathcal{H}(\eta, D)$ allows to reduce the water waves problem \eqref{S3:DimWholeLineProblem} to a scalar equation for $\eta:$
\begin{equation}\label{S3:Hequation1}
\partial_t\left(\mathcal{H}(\eta, D)\{ \eta_t\} \right) + \partial_x\left( \frac{1}{2}\left(\mathcal{H}(\eta, D)\{\eta_t\} \right)^2 + \epsilon \eta - \frac{1}{2} \frac{(\eta_t + \eta_x \mathcal{H}(\eta, D)\{ \eta_t\})^2}{1+\eta_x^2}\right) = 0,
\end{equation}
where the operator $H$ is described via 
\begin{equation}\label{S3:DimHbehav1}
\int_{-\infty}^{\infty} e^{-ikx}( i  \cosh(k(\eta+h)) f(x) - \sinh(k(\eta+h)) \mathcal{H}( \eta, D)\{ f(x) \}) \D x = 0, \qquad k \in \RR.
\end{equation}
In doing so, we went from having to solve for two unknowns $\phi$ and $\eta$ to having to solve for one unknown $\eta.$ Now, the utility of \eqref{S3:Hequation1} depends on whether we can find a useful representation of the operator $\mathcal{H}(\eta, D).$ 

\subsubsection{Perturbation expansion of the $\mathcal{H}$ operator}
In this section, we derive a representation of the $\mathcal{H}$ operator in the leading two terms. First, observe that \eqref{S3:DimHbehav1} is written in dimensional coordinates. To obtain the non-dimensional version, introduce new variables: 
\[ 
t^{\star} = \frac{t \sqrt{gh}}{L}, \qquad x^{\star}  = \frac{x}{L}, \qquad z^{\star}  = \frac{z}{h}, \qquad \eta^{\star}  = \frac{\eta}{a}, \qquad k^{\star}  = Lk, \]
rescale functions via
\[ \phi = \frac{Lga}{\sqrt{gh}} \phi^{\star}, \qquad q^{\star}  = \frac{\sqrt{gh}}{agL} q,\]
and define parameters $\epsilon$ and $\mu$ so that
\[ 
\epsilon = \frac{a}{h}, \qquad \mu = \frac{h}{L}, \qquad \epsilon \mu = \frac{a}{L}.
\] 
Starting with 
\begin{equation*}
\int_{\partial D} \varphi_z(\phi_z \D x - \phi_x \D z) + \phi(\varphi_{xx} \D x + \varphi_{xz} \D z) = 0,
\end{equation*}
one may continue the same procedure to obtain the nondimensional version of \eqref{S3:DimHbehav}:
\begin{equation}\label{S3:NDimHbehav}
\int_{-\infty}^{\infty} e^{-ikx}( i  \cosh(\mu k(\eta+1)) g(x) - \sinh(\mu k(\eta+1)) \mathcal{H}( \epsilon \eta, D)\{ g(x) \}) \D x = 0, \qquad k \in \RR.
\end{equation}
where we dropped the starred notation for convenience. In addition, note that $g$ and $f$ are related by 
\[ g(x^{\star}) = \frac{\sqrt{gh}}{ga}g(x^{\star}L) = \frac{\sqrt{gh}}{ga}f(x).
\]
Since $\epsilon \ll 1,$ we can expand the hyperbolic functions as a Taylor series in $\epsilon:$
\begin{align*}
\cosh(\mu k(\eta+1)) &= \cosh(\mu k) + \mu k \epsilon \eta \sinh(\mu k) + \mathcal{O}(\epsilon^2), \\
\sinh(\mu k(\eta+1)) &= \sinh(\mu k) + \mu k \epsilon \eta \cosh(\mu k) + \mathcal{O}(\epsilon^2).
\end{align*}
Now, we note that the idea of a regular perturbation series applies not only to classical functions but also to operators. Therefore, we expand
\begin{align*}
\mathcal{H}( \eta, D)\{ g(x) \} &= \left[\mathcal{H}_0 + \epsilon \mathcal{H}_1 + \mathcal{O}(\epsilon^2) \right]( \epsilon\eta, D)\{ g(x) \}.
\end{align*}
Equation \eqref{S3:NDimHbehav} becomes:
\begin{equation}\label{S3:HPerturbed}
\begin{aligned}
\int_{-\infty}^{\infty} &e^{-ikx}( i \left[ \cosh(\mu k) + \mu k \epsilon \eta \sinh(\mu k) + \mathcal{O}(\epsilon^2) \right] g(x) \\
&- \left[ \sinh(\mu k) + \mu k \epsilon \eta \cosh(\mu k) + \mathcal{O}(\epsilon^2)\right] \left[ \mathcal{H}_0 + \epsilon \mathcal{H}_1 + \mathcal{O}(\epsilon^2) \right]( \epsilon \eta, D)\{ g(x) \}) \D x = 0.
\end{aligned}
\end{equation}
\textbf{Within $\mathcal{O}(\epsilon^0):$} Using \eqref{S3:HPerturbed}, we obtain 
\begin{equation*}
\int_{-\infty}^{\infty} e^{-ikx}( i  \cosh(\mu k) g(x) - \sinh(\mu k) \mathcal{H}_0( \epsilon \eta, D)\{ g(x) \}) \D x = 0.
\end{equation*}
Dividing by $\sinh(\mu k)$ yields 
\begin{equation*}
\int_{-\infty}^{\infty} e^{-ikx}( i  \coth(\mu k) g(x) - \mathcal{H}_0( \epsilon \eta, D)\{ g(x) \}) \D x = 0.
\end{equation*}
Splitting the integrand and recognizing the Fourier transform yields:
\begin{align*}
\mathcal{F}_k\{\mathcal{H}_0( \epsilon \eta, D)\{ g(x) \}\}&= \int_{-\infty}^{\infty} e^{-ikx}\mathcal{H}_0( \epsilon \eta, D)\{ g(x) \} \D x\\
&= \int_{-\infty}^{\infty} e^{-ikx} i \coth(\mu k) g(x) \D x \\
&= i \coth(\mu k) \mathcal{F}_k\{g(x)\}.
\end{align*}
Finally, we invert Fourier transform to obtain 
\begin{equation}\label{S3:H0}
\mathcal{H}_0( \epsilon \eta, D)\{ g(x) \} =\mathcal{F}^{-1}_k\{i \coth(\mu k) \mathcal{F}_k\{g(x)\} \},
\end{equation}
where we write out $k$'s explicitly to keep track of transforms. 
\newline \textbf{Within $\mathcal{O}(\epsilon^1):$} from \eqref{S3:HPerturbed}, we obtain
\begin{align*}
\int_{-\infty}^{\infty} e^{-ikx}( i \mu k \eta \sinh(\mu k) g(x) - \left[ \sinh(\mu k)\mathcal{H}_1 + \mu k \eta \cosh(\mu k) \mathcal{H}_0 \right]( \epsilon \eta, D)\{ g(x) \}) \D x = 0.
\end{align*}
Dividing by $\sinh(\mu k)$ and dropping $( \epsilon \eta, D)$ for the notational convenience, we have 
\begin{align*}
\int_{-\infty}^{\infty} e^{-ikx}( i \mu k \eta g(x) - \left[\mathcal{H}_1 + \mu k \eta \coth(\mu k) \mathcal{H}_0 \right]\{ g(x) \}) \D x = 0.
\end{align*}
Splitting the integral and recognising the Fourier transform yields:
\begin{align*}
\mathcal{F}_k \{\mathcal{H}_1 \{ g(x) \})\} &= \int_{-\infty}^{\infty} e^{-ikx}  \mathcal{H}_1 \{ g(x) \}) \D x \\
&= \int_{-\infty}^{\infty} e^{-ikx}( i \mu k \eta g  - \mu k \eta \coth(\mu k) \mathcal{H}_0 \{ g(x) \}) \D x \\
&= \mu \mathcal{F}_k \{ i k \eta g \} - \mu k \coth(\mu k) \mathcal{F}_k\{ \eta \mathcal{H}_0 \{ g(x) \}) \}.
\end{align*}
Inverting Fourier transform and using \eqref{S3:H0}, we obtain an expression for $\mathcal{H}_1:$
\begin{align*}
\mathcal{H}_1 \{ g(x) \} &= \mathcal{F}^{-1}_k \{ \mu \eta g \} - \mathcal{F}^{-1}_k \{ \mu k \coth(\mu k) \mathcal{F}_k \{ \eta \mathcal{H}_0 \{ g(x) \}) \} \} \\
&= \mu \partial_x(\eta g) - \mathcal{F}^{-1}_k \{ \mu k \coth(\mu k) \mathcal{F}_k \{ \eta \mathcal{H}_0 \{ g(x) \}) \} \} \\
&= \mu \partial_x(\eta g) - \mathcal{F}^{-1}_k \{ \mu k \coth(\mu k) \mathcal{F}_k \{ \eta \mathcal{F}^{-1}_l \{i \coth(\mu l) \mathcal{F}_l\{g\} \} \} \}.
\end{align*}
In sum, we find a representation for the $\mathcal{H}$ operator within two orders:
\[ \mathcal{H}( \epsilon \eta, D)\{ g(x) \} = [ \mathcal{H}_0 + \epsilon \mathcal{H}_1]( \epsilon \eta, D)\{ g(x) \} + \mathcal{O}(\epsilon^2),\]
where 
\begin{align*}
\mathcal{H}_0( \epsilon \eta, D)\{ g(x) \} &= \mathcal{F}^{-1}_k\{i \coth(\mu k) \mathcal{F}_k\{g(x)\} \}, \\
\mathcal{H}_1( \epsilon \eta, D) \{ g(x) \} &= \mu \partial_x(\eta g) - \mathcal{F}^{-1}_k \{ \mu k \coth(\mu k) \mathcal{F}_k \{ \eta \mathcal{F}^{-1}_l\{i \coth(\mu l) \mathcal{F}_l\{g\}\} \} \}.
\end{align*}
\rmk{As we proceed to use the operator $\mathcal{H},$ we must exercise caution. Recall the expression for $\mathcal{H}_0:$
\[ \mathcal{H}_0( \epsilon \eta, D)\{ g(x) \} =\mathcal{F}^{-1}_k\{i \coth(\mu k) \mathcal{F}_k\{g(x)\} \}. \]
Expanding $\coth(\mu k)$ via its Laurent series in $\mu k$ gives
\[ \coth(\mu k) = \frac{1}{\mu k} +\frac{\mu k}{3} + \mathcal{O}(\mu^3).\]
It is readily seen that since $\coth(\mu k)$ has a simple pole as $k \to 0,$ so do $\mathcal{H}_0, \mathcal{H}_1$ and $\mathcal{H}.$ 
}
\subsubsection{Deriving an expression for surface elevation: the whole-line.}
Having found an expression for the $\mathcal{H}$ operator, we proceed to derive approximate equations for the surface $\eta.$ Recall the scalar equation \eqref{S3:Hequation}: 
\begin{equation}\label{S3:Eq12}
\partial_t\left(\mathcal{H}(\eta, D)\{ \eta_t\} \right) + \partial_x\left( \frac{1}{2}\left(\mathcal{H}(\eta, D)\{\eta_t\} \right)^2 + \epsilon \eta - \frac{1}{2} \frac{(\eta_t + \eta_x \mathcal{H}(\eta, D)\{ \eta_t\})^2}{1+\eta_x^2}\right) = 0.
\end{equation}
The non-dimensional version of \eqref{S3:Eq12} is given by
\begin{equation}\label{S3:Eq13}
\partial_t\left(\mathcal{H}(\epsilon\eta, D)\{ \epsilon \mu \eta_t\} \right) + \partial_x\left( \frac{1}{2}\left(\mathcal{H}(\epsilon\eta, D)\{ \epsilon \mu \eta_t\} \right)^2 + \epsilon \eta - \frac{1}{2}\epsilon^2 \mu^2 \frac{(\eta_t + \eta_x \mathcal{H}(\epsilon\eta, D)\{ \epsilon \mu \eta_t\})^2}{1+\epsilon^2 \mu^2 \eta_x^2}\right) = 0.
\end{equation}
Also, recall our maximal balance assumption $\epsilon = \mu^2,$ and recall the first-order and second order expansions for $\coth(\mu k):$
\[ 
\coth(\mu k) = \frac{1}{\mu k} + \mathcal{O}(\mu) = \frac{1}{\mu k} + \frac{\mu k}{3}+ \mathcal{O}(\mu^3).
 \]
\newline \textbf{Within $\mathcal{O}(\mu^0):$} In the leading order, the equation \eqref{S3:Eq13} becomes
\[ 
\partial_t\left(\mathcal{H}_0(\epsilon\eta, D)\{ \epsilon \mu \eta_t\} \right) + \epsilon \partial_x \eta = 0.
\]
Bringing $\partial_t$ inside $\mathcal{H}_0$ and substituting the expression for $\mathcal{H}_0,$ we obtain:
\[ 
\mathcal{F}^{-1}_k \{i \coth(\mu k) \mathcal{F}_k \{\epsilon \mu \eta_{tt}\} \}+ \epsilon \partial_x \eta = 0.
\]
Inverting the Fourier transform and multiplying by $k/(i \epsilon)$ yields
\[ 
\mu k \coth(\mu k) \reallywidehat{\eta_{tt}}_k  +  k^2 \reallywidehat{\eta}_k = 0.
\]
Expanding $\coth(\mu k)$ in the leading order gives
\[ 
\reallywidehat{\eta_{tt}}_k + k^2 \reallywidehat{\eta}_k = 0.
\]
Inverting the Fourier transform, we have
\[ 
\eta_{tt} + (-i \partial_x)^2 \eta = 0,
\]
which is
\[ \eta_{tt} - \eta_{xx} = 0. \]
This is the wave equation, as we desired. 
\vspace{5mm}
\newline \textbf{Within $\mathcal{O}(\mu^2):$} We proceed to derive the approximate equations in the next leading order. The non-dimensional equation \eqref{S3:Eq13} becomes
\begin{equation}\label{S3:Eq14}
\partial_t\left(\mathcal{H}_0 \{ \epsilon \mu \eta_t\} + \epsilon \mathcal{H}_1 \{ \epsilon \mu \eta_t\} \right) + \partial_x \left(\frac{1}{2} (\mathcal{H}_0 \{ \epsilon \mu \eta_t\})^2 + \epsilon \eta \right) = 0.
\end{equation}
Recall
\begin{align*}
\mathcal{H}_0(\epsilon\eta, D)\{ \epsilon \mu \eta_t \} &= \epsilon \mu \mathcal{F}^{-1}_k \{ i \coth (\mu k) \reallywidehat{\eta_t}_k\}; \\
\mathcal{H}_1(\epsilon\eta, D)\{ \epsilon \mu \eta_t \} &= \epsilon \mu^2 (\eta \eta_t)_x - \epsilon \mathcal{F}^{-1}_k \{ \mu k \coth (\mu k) \mathcal{F}_k\{\eta \mathcal{F}^{-1}_l \{ i  \mu\coth (\mu l) \reallywidehat{\eta_t}_l\} \}\}.
\end{align*}
Observe that
\begin{align*}
\frac{1}{2}\left(\mathcal{H}_0(\epsilon\eta, D)\{ \epsilon \mu \eta_t\} \right)^2 = \frac{1}{2}\left(\mathcal{H}_0(\epsilon\eta, D)\{ \epsilon \mu \eta_t\} \right)^2 = \frac{\epsilon^2}{2} \left( \mathcal{F}^{-1}_k \{ i \mu \coth (\mu j) \reallywidehat{\eta_t}_k\}\right)^2,
\end{align*}
and 
\begin{align*}
\partial_t\left(\left[ \mathcal{H}_0(\epsilon\eta, D) + \epsilon \mathcal{H}_1(\epsilon\eta, D) \right] \{ \epsilon \mu \eta_t\} \right) &= \epsilon \mu \mathcal{F}^{-1}_k \{ i \coth (\mu k) \reallywidehat{\eta_{tt}}_k\} + \epsilon^2 \mu^2 (\eta \eta_t)_{tx} \\
&- \epsilon^2 \mathcal{F}^{-1}_k \{ \mu k \coth (\mu k) \mathcal{F}_k\{\partial_t \left[\eta \mathcal{F}^{-1}_l \{ i \mu \coth (\mu l) \reallywidehat{\eta_t}_l\} \right] \}\}.
\end{align*}
The single equation \eqref{S3:Eq14} becomes 
\begin{align*}
\epsilon \mu \mathcal{F}^{-1}_k \{ i \coth (\mu k) \reallywidehat{\eta_{tt}}_k\} + \epsilon^2 \mu^2 (\eta \eta_t)_{tx} &- \epsilon^2 \mathcal{F}^{-1}_k \{ \mu k \coth (\mu k) \mathcal{F}_k\{\partial_t \left[\eta \mathcal{F}^{-1}_l \{ i \mu \coth (\mu l) \reallywidehat{\eta_t}_l\} \right] \} \} \\
&+\frac{\epsilon^2}{2} \partial_x\left( \mathcal{F}^{-1}_j \{ i \mu \coth (\mu j) \reallywidehat{\eta_t}_j\}\right)^2 + \epsilon \partial_x \eta = 0.
\end{align*}
Application of Fourier transform yields
\begin{align*}
\epsilon \mu i \coth (\mu k) \reallywidehat{\eta_{tt}}_k  + \epsilon^2 \mu^2 ik (\eta \eta_t)_{t} &- \epsilon^2 \mu k \coth (\mu k) \mathcal{F}_k\{\partial_t \left[\eta \mathcal{F}^{-1}_l \{ i \mu \coth (\mu l) \reallywidehat{\eta_t}_l\} \right] \} \\
&+\frac{\epsilon^2}{2} ik \mathcal{F}_k \{\left( \mathcal{F}^{-1}_j \{ i \mu \coth (\mu j) \reallywidehat{\eta_t}_j\}\right)^2\} + \epsilon ik \reallywidehat{\eta}_k = 0.
\end{align*}
Divide by $i\epsilon$ and recall $\epsilon = \mu^2$ to obtain
\begin{align*}
\mu \coth (\mu k) \reallywidehat{\eta_{tt}}_k  +  \mu^4 k (\eta \eta_t)_{t} &- \mu^3 k \coth (\mu k) \mathcal{F}\{\partial_t \left[\eta \mathcal{F}^{-1} \{ \mu \coth (\mu l) \reallywidehat{\eta_t}_l\}_l \right] \}_k \\
&+\frac{\mu^2}{2} k \mathcal{F}\{\left( \mathcal{F}^{-1} \{ i \mu \coth (\mu j) \reallywidehat{\eta_t}_j\}_j\right)^2\}_k + k \reallywidehat{\eta}_k = 0.
\end{align*}
Expanding $\coth(\mu k)$-like terms yields
\begin{equation}\label{S3:Eq15}
\begin{aligned}
\left( \frac{1}{k} + \frac{\mu^2 k}{3} \right) \reallywidehat{\eta_{tt}}_k  + \mu^4 k (\eta \eta_t)_{t} &- \mu^2 k \left( \frac{1}{k} + \frac{\mu^2 k}{3} \right) \mathcal{F}_k\{\partial_t \left[\eta \mathcal{F}^{-1} \{\left( \frac{1}{l} + \frac{\mu^2 l}{3}\right) \reallywidehat{\eta_t}_l\}_l \right] \} \\
&-\frac{\mu^2}{2} k \mathcal{F}_k \{\left( \mathcal{F}^{-1}_j \{ \left( \frac{1}{j} + \frac{\mu^2 j}{3} \right) \reallywidehat{\eta_t}_j\}\right)^2\} + k \reallywidehat{\eta}_k = 0.
\end{aligned}
\end{equation}
Within $\mathcal{O}(\mu^2),$ the equation \eqref{S3:Eq15} becomes 
\begin{align*}
\left( \frac{1}{k} + \frac{\mu^2 k}{3} \right) \reallywidehat{\eta_{tt}}_k - \mu^2 \mathcal{F}_k\{\partial_t \left[\eta \mathcal{F}^{-1}_l \{\frac{1}{l}\reallywidehat{\eta_t}_l\} \right] \} -\frac{\mu^2}{2} k \mathcal{F}_k\{\left( \mathcal{F}^{-1}_j \{ \frac{1}{j} \reallywidehat{\eta_t}_j\}\right)^2\} + k \reallywidehat{\eta}_k = 0,
\end{align*}
or rearranging and multiplying by $k$, we have
\begin{align*}
\reallywidehat{\eta_{tt}}_k + k^2 \reallywidehat{\eta}_k + \mu^2 \left(\frac{ k^2}{3}\reallywidehat{\eta_{tt}}_k - k \mathcal{F}_k\{\partial_t \left[\eta \mathcal{F}^{-1}_l \{\frac{1}{l}\reallywidehat{\eta_t}_l\} \right] \} -\frac{1}{2} k^2 \mathcal{F}_k \{\left( \mathcal{F}^{-1}_j \{ \frac{1}{j} \reallywidehat{\eta_t}_j\}\right)^2\} \right) = 0.
\end{align*}
Finally, inverting the Fourier transform yields:
\begin{align*}
\eta_{tt} - \eta_{xx} + \mu^2 \left(-\frac{\partial_x^2}{3}\eta_{tt} + i \partial_x\left(\partial_t \left[\eta \mathcal{F}^{-1}_l \{\frac{1}{l}\reallywidehat{\eta_t}_l\} \right] \right) + \frac{1}{2} \partial_x^2 \left( \mathcal{F}^{-1}_j \{ \frac{1}{j} \reallywidehat{\eta_t}_j\}\right)^2 \right) = 0,
\end{align*}
or more conveniently,
\begin{equation}\label{S3:Eq16}
\eta_{tt} - \eta_{xx} = \mu^2 \left(\frac{\partial_x^2}{3}\eta_{tt} - i \partial_x\left(\partial_t \left[\eta \mathcal{F}^{-1}_l \{\frac{1}{l}\reallywidehat{\eta_t}_l\} \right] \right) - \frac{1}{2} \partial_x^2 \left( \mathcal{F}^{-1}_j \{ \frac{1}{j} \reallywidehat{\eta_t}_j\}\right)^2 \right).
\end{equation}
We seek to simplify \eqref{S3:Eq16}. First, integration by parts gives
\begin{align*}
\frac{1}{l} \reallywidehat{\eta_t}_l &= \frac{1}{l} \frac{2}{\pi}\int^{\infty}_{-\infty} e^{-ilx} \eta_t \D x \\
&= \frac{1}{l} \frac{2}{\pi} e^{-ilx} \int^x_{-\infty} \eta_t(x', t) \D x' \bigg|^{\infty}_{-\infty} + i\frac{2}{\pi} \int^{\infty}_{-\infty}e^{-ilx} \int^x_{-\infty} \eta_t(x', t) \D x' \D x  \\
&= i\frac{2}{\pi} \int^{\infty}_{-\infty}e^{-ilx} \int^x_{-\infty} \eta_t(x', t) \D x' \D x \\
&= i \mathcal{F}_l\{\int^x_{-\infty} \eta_t(x', t) \D x' \},
\end{align*}
so that 
\begin{equation}\label{S3:Eq17}
\mathcal{F}^{-1}_l \{\frac{1}{l} \reallywidehat{\eta_t}_l\} = \mathcal{F}^{-1}_l \{  i \mathcal{F}_l\{\int^x_{-\infty} \eta_t(x', t) \D x' \} \} = i \int^x_{-\infty} \eta_t(x', t) \D x',
\end{equation}
where we applied the Fourier inversion theorem. Using \eqref{S3:Eq17} and that $\eta_{tt} = \eta_{xx} + \mathcal{O}(\mu^2),$ the equation \eqref{S3:Eq16} becomes
\begin{align}
\eta_{tt} - \eta_{xx}  &= \mu^2 \left(\frac{1}{3}\eta_{xxxx} + \partial_x\partial_t \left[\eta \left(\int^{x}_{-\infty} \eta_t \D x' \right) \right]  + \frac{1}{2} \partial_x^2 \left(\int^{x}_{-\infty} \eta_t \D x' \right)^2 \right) \nonumber \\
&= \mu^2 \left( \frac{1}{3}\eta_{xxxx} + \partial_x \left[ \eta_t \left(\int^{x}_{-\infty} \eta_t \D x'  \right) + \eta \eta_x\right]  + \frac{1}{2} \partial_x^2 \left(\int^{x}_{-\infty} \eta_t \D x' \right)^2 \right) \nonumber \\
&= \epsilon \left[ \frac{1}{3}\eta_{xxxx} +  \partial_x^2 \left( \frac{\eta^2}{2} + \left( \int^{x}_{-\infty} \eta_t \D x' \right)^2\right)\right] \label{S3:Eq18}
\end{align}
In summary, the second order approximation of a scalar equation for $\eta$ resulted into equation \eqref{S3:Eq18}.
\rmk{At the end of Section 2.4, we mentioned secular terms in the next order. We now show directly by examining the dispersion relation \eqref{S3:Eq18}. Assume a plane wave solution of the form $\tilde{\eta}(x,t) = \exp(i(kx-\omega t)).$ Substituting $\tilde{\eta}$ into the linearised equation
\[ 
\eta_{tt} - \eta_{xx} = \epsilon \frac{1}{3}\eta_{xxxx},
\]
leads to the following relation
\[ -\omega^2 + k^2 = \epsilon \frac{k^4}{3}.\]
Substituting the negative root of $\omega$ into the wave solution gives
\[ \eta(x,t) \approx \exp(ikx) \exp\left( \sqrt{\frac{\epsilon}{3}}k^2 t\right).\]
As is seen, this solution is unbounded in time as $k \to \infty.$ This phenomenon suggests that \eqref{Eq18} contains secularity and therefore warrants an application of time scales. 
}

\subsubsection{Derivation of wave and KdV equations: the whole line.}
We derive the approximate equations from
\begin{equation}\label{srfceq0}
\eta_{tt} - \eta_{xx} = \epsilon \left[ \frac{1}{3}\eta_{xxxx} +  \partial_x^2 \left( \frac{\eta^2}{2} + \left( \int^{x}_{-\infty} \eta_t \D x' \right)^2\right)\right].
\end{equation}
We assume an expansion of $\eta$ in $\epsilon:$
\begin{equation}\label{SrfcExpansion}
\eta = \eta_0 + \epsilon \eta_1 + \mathcal{O}(\epsilon^2).
\end{equation}
\subsubsection*{First order approximation}
Substitution of \eqref{SrfcExpansion} into equation \eqref{srfceq0} yields
\begin{equation}\label{srfceqExpanded}
\begin{aligned}
\eta_{0tt} - \eta_{0xx} +&\epsilon(\eta_{1tt} - \eta_{1xx}) \\
&= \epsilon \left[ \frac{1}{3}\eta_{0xxxx} +  \partial_x^2 \left( \frac{(\eta_0 + \epsilon \eta_1)^2}{2} + \left( \int^{x}_{-\infty} (\eta_0 + \epsilon \eta_1)_t \D x' \right)^2\right)\right] + \mathcal{O}(\epsilon^2). 
\end{aligned}
\end{equation}
In the leading order $\mathcal{O}(\epsilon^0),$ equation \eqref{srfceqExpanded} becomes
\begin{equation*}
\eta_{0tt} - \eta_{0xx} = 0.
\end{equation*}
This is the wave equation with velocity $1,$ and whose general solution is 
\[ \eta_0 = F(x-t) + G(x+t), \]
where $F,G$ are general functions. 
\subsubsection*{Second order approximation}
As was discussed in Remark 5, anticipating the secular terms, we introduce slow time scales
\[ \tau_0 = t, \qquad \tau_1 = \epsilon t. \]
so that 
\[ \eta(x, t) = \eta(x, \tau_0, \tau_1). \]
The expansion \eqref{SrfcExpansion} becomes
\begin{equation}\label{NewSrfcExpansion}
\eta(x, \tau_0, \tau_1) = \eta_0(x, \tau_0, \tau_1) + \mathcal{O}(\epsilon^1).
\end{equation}
Substituting \eqref{NewSrfcExpansion} into \eqref{srfceq0}, within $\mathcal{O}(\epsilon^0),$ we obtain
\begin{equation}\label{1stOrderApprox}
\eta_{0\tau_0 \tau_0} - \eta_{0xx} = 0,
\end{equation}
so that the general solution is 
\[ \eta_0(x, \tau_0, \tau_1) = F(x-\tau_0, \tau_1) + G(x+\tau_0, \tau_1). \]
Although we have found an expression for $\eta_0,$ the functions $F,G$ used are still general functions. To determine them, we proceed to the next order, where we can understand the dependence of $F,G$ on the slow time scale $\tau_1.$ In addition, we introduce left-going and right-going variables 
\[ 
\xi = x-\tau_0 \qquad \zeta = x+ \tau_0.
\]
These new variables imply
\begin{align*}
\partial_x &= \partial_\xi \frac{\D \xi}{\D x} + \partial_\zeta \frac{\D \zeta}{\D x} =\partial_\xi + \partial_\zeta, \\
\partial_t &= \partial_\xi \frac{\D \xi}{\D t} + \partial_\zeta \frac{\D \zeta}{\D t} + \partial_{\tau_1}\frac{\D \tau_1}{\D t} = - \partial_\xi + \partial_\zeta + \epsilon \partial_{\tau_1}.
\end{align*}
We can rewrite \eqref{NewSrfcExpansion} as follows:
\begin{align*}
\eta &= \eta_0 + \epsilon \eta_1 + \mathcal{O}(\epsilon^2)  \\
&= F(\xi, \tau_1) + G(\zeta, \tau_1) + \epsilon \eta_1 + \mathcal{O}(\epsilon^2).
\end{align*}
For ease of writing, we suppress explicit dependence on variables, though the reader should bear in mind that function $F ~ (G)$ depend on $\xi ~ (\zeta), \tau_1.$ Observe that
\begin{align*}
(\partial_t^2 - \partial_x^2) &=  \left( - 4\partial_\xi \partial_\zeta + 2\epsilon(\partial_\zeta \partial_{\tau_1} - \partial_\xi\partial_{\tau_1}) + \epsilon^2 \partial_{\tau_1}^2 \right),
\end{align*}
%\begin{align*}
%(\partial_t^2 - \partial_x^2) &= \left( (- \partial_\xi + \partial_\zeta + \epsilon \partial_{\tau_1})^2 - (\partial_\xi + \partial_\zeta)^2 \right) \\
%&= \left( \partial^2_\xi - 2\partial_\xi\partial_\zeta + \partial_\zeta^2 + 2\epsilon(\partial_\zeta \partial_{\tau_1} - \partial_\xi\partial_{\tau_1}) + \epsilon^2 \partial_{\tau_1}^2
%- \partial_\xi^2 - 2\partial_\xi\partial_\zeta - \partial_\zeta^2 \right) \\
%&= \left( - 4\partial_\xi \partial_\zeta + 2\epsilon(\partial_\zeta \partial_{\tau_1} - \partial_\xi\partial_{\tau_1}) + \epsilon^2 \partial_{\tau_1}^2 \right),
%\end{align*}
so that the LHS of \eqref{srfceq0} becomes
\begin{equation}
(\partial_t^2 - \partial_x^2) \eta =  \epsilon \left(- 4\eta_{1\xi \zeta} - 2F_{\tau_1 \xi} + 2G_{\tau_1 \zeta} \right) + \mathcal{O}(\epsilon^2). \label{LHS1}
\end{equation}
%\begin{align}
%(\partial_t^2 - \partial_x^2) \eta
%&= \left( - 4\partial_\xi \partial_\zeta + 2\epsilon(\partial_\zeta \partial_{\tau_1} - \partial_\xi\partial_{\tau_1}) + \epsilon^2 \partial_{\tau_1}^2 \right) ( F + G + \epsilon \eta_1 + \mathcal{O}(\epsilon^2)) \nonumber \\
%&= - 4\partial_\xi \partial_\zeta  ( F + G + \epsilon \eta_1) + 2\epsilon(\partial_\zeta \partial_{\tau_1} - \partial_\xi\partial_{\tau_1})  ( F + G) + \mathcal{O}(\epsilon^2) \nonumber \\
%&=  \epsilon \left(- 4\eta_{1\xi \zeta} - 2F_{\tau_1 \xi} + 2G_{\tau_1 \zeta} \right) + \mathcal{O}(\epsilon^2). \label{LHS1}
%\end{align}
Now, we deal with the RHS of \eqref{srfceq0}. By appropriate substitutions, the terms become:
\begin{align*}
\frac{1}{3}\eta_{xxxx} &= \frac{1}{3}(F_{\xi\xi\xi\xi} + G_{\zeta\zeta\zeta\zeta}) + \mathcal{O}(\epsilon); \\
\frac{1}{2}\eta^2 &= \frac{1}{2} (F^2 + 2FG +G^2) + \mathcal{O}(\epsilon);
\end{align*}
and 
\begin{align*}
\left( \int^{x}_{-\infty} \eta_t \D x' \right)^2 &= \left( \int^{x}_{-\infty} \eta_{0t} \D x' \right)^2 + \mathcal{O}(\epsilon) \\
&= \left( \int^{x}_{-\infty}-F_\xi + G_\zeta \D x' \right)^2 + \mathcal{O}(\epsilon)\\
&= \left( \int^{x}_{-\infty}F_\xi \D x'\right)^2 - 2\left( \int^{x}_{-\infty}F_\xi \D x'\right)\left( \int^{x}_{-\infty}G_\zeta \D x'\right) + \left( \int^{x}_{-\infty} G_\zeta \D x'\right)^2 + \mathcal{O}(\epsilon) \\
&= F^2 - 2FG + G^2 + \mathcal{O}(\epsilon),
\end{align*}
%\begin{align*}
%\frac{1}{3}\eta_{xxxx} &=\frac{1}{3} (\partial_x^2)^2 \eta \\
%&=\frac{1}{3} (\partial_\xi^2 + 2\partial_\xi\partial_\zeta + \partial_\zeta^2 )^2 \eta \\
%&=\frac{1}{3} (\partial_\xi^4 + \partial_\zeta^4 +  4\partial_\xi^3\partial_\zeta+  2\partial_\xi\partial_\zeta^3 + 6\partial_\xi\partial_\zeta) (F + G + \epsilon \eta_1 + \mathcal{O}(\epsilon^2)) \\
%&= \frac{1}{3}(F_{\xi\xi\xi\xi} + G_{\zeta\zeta\zeta\zeta} + \epsilon (\partial_\xi + \partial_\zeta)^4 \eta_1 + \mathcal{O}(\epsilon^2)) \\
%&= \frac{1}{3}(F_{\xi\xi\xi\xi} + G_{\zeta\zeta\zeta\zeta} + \mathcal{O}(\epsilon)); \\
%\frac{1}{2}\eta^2 &= \frac{1}{2} \left( F+G + \epsilon \eta_1 \right)^2 \\
%&=  \frac{1}{2} \left( (F+G)^2 + 2\epsilon(F+G)\eta_1 + \epsilon^2 \eta_1^2 \right) \\ &=  \frac{1}{2} (F^2 + 2FG +G^2) + \epsilon(F+G)\eta_1 + \mathcal{O}(\epsilon^2) \\
%&= \frac{1}{2} (F^2 + 2FG +G^2) + \mathcal{O}(\epsilon); \\ \left( \int^{x}_{-\infty} \eta_t \D x' \right)^2 &= \left( \int^{x}_{-\infty} \eta_{0t} \D x' + \epsilon\int^{x}_{-\infty}  \eta_{1t} \D x' \right)^2 \\
%&= \left( \int^{x}_{-\infty} \eta_{0t} \D x' + \epsilon\int^{x}_{-\infty} \eta_{1t} \D x' \right)^2 \\ &= \left( \int^{x}_{-\infty} \eta_{0t} \D x' \right)^2 + \mathcal{O}(\epsilon) \\
% &= \left( \int^{x}_{-\infty}(-\partial_\xi + \partial_\zeta + \epsilon \partial_{\tau_1}) (F+G) \D x' \right)^2 + \mathcal{O}(\epsilon)\\  &=  \left( \int^{x}_{-\infty}-F_\xi + G_\zeta \D x' + \epsilon \int^{x}_{-\infty} \partial_{\tau_1} (F+G) \D x' \right)^2 + \mathcal{O}(\epsilon) \\
%&= \left( \int^{x}_{-\infty}-F_\xi + G_\zeta \D x' \right)^2 + \mathcal{O}(\epsilon)\\ &= \left( \int^{x}_{-\infty}F_\xi \D x'\right)^2 - 2\left( \int^{x}_{-\infty}F_\xi \D x'\right)\left( \int^{x}_{-\infty}G_\zeta \D x'\right) + \left( \int^{x}_{-\infty} G_\zeta \D x'\right)^2 + \mathcal{O}(\epsilon) \\
%&= F^2 - 2FG + G^2 + \mathcal{O}(\epsilon), \end{align*}
where we assume that $F,G$ vanish as $\xi, \zeta \to -\infty.$
%\begin{align*}
%\int^{x}_{-\infty}F_\xi \D x' = \lim_{a\to -\infty} \int^{x}_{a}F_{\xi'}(x'-t, \tau_1) \D x' &=  \lim_{a\to -\infty} \int^{x-t}_{a-t}F_{\xi'}(\xi', \tau_1) \D \xi' \\
%&=  \lim_{a\to -\infty} \int^{\xi}_{a-t}F_{\xi'}(\xi', \tau_1) \D \xi' \\
%&= \int^{\xi}_{-\infty}F_{\xi'}(\xi', \tau_1) \D \xi' = F(\xi, \tau_1), \\
%\int^{x}_{-\infty} G_\zeta' \D x' = \lim_{a\to -\infty} \int^{x}_{a}F_{\zeta'}(x'-t, \tau_1) \D x' &=  \lim_{a\to -\infty} \int^{x+t}_{a+t}G_{\zeta'}(\zeta', \tau_1) \D \zeta' \\
%&=  \lim_{a\to -\infty} \int^{\zeta}_{a-t}G_{\zeta'}(\zeta', \tau_1) \D \zeta' \\
%&= \int^{\zeta}_{-\infty}G_{\zeta'}(\zeta', \tau_1) \D \zeta' = G(\zeta, \tau_1).
%\end{align*}
The RHS of \eqref{srfceq0} becomes
\begin{align}
\epsilon &\left[ \frac{1}{3}\eta_{xxxx} +  \partial_x^2 \left( \frac{\eta^2}{2} + \left( \int^{x}_{-\infty} \eta_t \D x' \right)^2\right)\right] \nonumber\\
&=\epsilon \Bigg[ \frac{1}{3}(F_{\xi\xi\xi\xi} + G_{\zeta\zeta\zeta\zeta}) +  (\partial_\xi^2 + 2\partial_\xi \partial_\zeta + \partial_\zeta^2) \left(\frac{1}{2} (F^2 + 2FG +G^2) + F^2 - 2FG + G^2 \right)\Bigg] + \mathcal{O}(\epsilon^2) \nonumber \\
&= \epsilon \Bigg[ \frac{1}{3}(F_{\xi\xi\xi\xi} + G_{\zeta\zeta\zeta\zeta}) +  (\partial_\xi^2 + 2\partial_\xi \partial_\zeta + \partial_\zeta^2) \left(\frac{3}{2} F^2  + \frac{3}{2}G^2 - FG \right) \Bigg] + \mathcal{O}(\epsilon^2). \label{RHS1}
\end{align}
Combining \eqref{LHS1} and \eqref{RHS1}, in $\mathcal{O}(\epsilon^1)$ we have
\begin{equation}\label{srfceq2}
- 4\eta_{1\xi \zeta} = 2F_{\tau_1 \xi} - 2G_{\tau_1 \zeta} + \frac{1}{3}(F_{\xi\xi\xi\xi} + G_{\zeta\zeta\zeta\zeta}) + (\partial_\xi^2 + 2\partial_\xi \partial_\zeta + \partial_\zeta^2) \left(\frac{3}{2} F^2  + \frac{3}{2}G^2 - FG \right).
\end{equation}
In the last term of \eqref{srfceq2}, differentiation yields:
\begin{align*}
(\partial_\xi^2 + 2\partial_\xi \partial_\zeta + \partial_\zeta^2) \left(\frac{3}{2} F^2  + \frac{3}{2}G^2 - FG\right) &= \partial_\xi(3 F F_\xi - G F_\xi) + \partial_\zeta(3 G G_\zeta - F G_\zeta) - 2 F_\xi G_\zeta,
\end{align*}
so that equation \eqref{srfceq2} becomes
\begin{align}
- 4\eta_{1\xi \zeta} &= 2F_{\tau_1 \xi} - 2G_{\tau_1 \zeta} + \frac{1}{3}(F_{\xi\xi\xi\xi} + G_{\zeta\zeta\zeta\zeta}) + \partial_\xi(3 F F_\xi - G F_\xi) + \partial_\zeta(3 G G_\zeta - F G_\zeta) - 2 F_\xi G_\zeta \nonumber \\
&= \partial_\xi(2F_{\tau_1} + \frac{1}{3}F_{\xi\xi\xi} + 3 F F_\xi) + \partial_\zeta(- 2G_{\tau_1} +  \frac{1}{3}G_{\zeta\zeta\zeta} + 3 G G_\zeta) - (G F_\xi  + F G_\zeta) - 2 F_\xi G_\zeta. \label{srfceq3}
\end{align}
Integration of \eqref{srfceq3} with respect to $\zeta$ yields
\[ 
- 4\eta_{1\xi} = \partial_\xi(2F_{\tau_1} + \frac{1}{3}F_{\xi\xi\xi} + 3 F F_\xi) \zeta + (- 2G_{\tau_1} +  \frac{1}{3}G_{\zeta\zeta\zeta} + 3 G G_\zeta) - \left(F_\xi \int G \D \zeta   + G F\right),
\]
and further integration with respect to $\xi$ leads to
\begin{equation}\label{Eq18}
- 4\eta_{1} = (2F_{\tau_1} + \frac{1}{3}F_{\xi\xi\xi} + 3 F F_\xi) \zeta + (- 2G_{\tau_1} +  \frac{1}{3}G_{\zeta\zeta\zeta}+ 3 G G_\zeta) \xi - \left(F \int G \D \zeta  + G \int F \D \xi \right).
\end{equation}
From the equation \eqref{Eq18}, we see that the $\xi, \zeta$ terms are the secular terms we wish to remove. Otherwise, $\eta_1$ is unbounded in time. Therefore, we must have 
\begin{align}
2F_{\tau_1} + \frac{1}{3}F_{\xi\xi\xi} + 3 F F_\xi &= 0 \label{KdV1} \\
2G_{\tau_1} - \frac{1}{3}G_{\zeta\zeta\zeta} -  3 G G_\zeta &= 0. \label{KdV2}
\end{align}
The equations \eqref{KdV1} and \eqref{KdV2} are KdV equations, which allow us to determine the behaviour of $F, G$ on a slow time scale $\tau_1.$ 

In conclusion, asymptotic analysis of the non-local formulation in shallow water limit gives rise to two KdV equations, \eqref{KdV1} and \eqref{KdV2}, for the right-going and left-going waves. Given the right decay initial conditions, we can solve these PDEs by means of the inverse scattering transform (see \cite[Chapter 9]{Ablowitz}). Keeping the leading-order terms, we have an approximate solution for the surface variable
\[ \eta \approx \eta_0 = F(x- t, \epsilon t) + G(x + t, \epsilon t).\]
The derivation is complete.

\section{Discussion}
In this section, we discuss and justify the relevance of the derivation. Taking the water wave problem on the whole line as our starting point, we obtain wave and KdV equations in the shallow water limit. This result is not novel; indeed, the AFM formulation, given in \cite{AFM2006}, also arrives at the same equations, though the unknown variable there is $q(x) = \phi(x,\eta(x)),$ the velocity potential evaluated at the surface. Rather, what is novel by our result is that we have derived the expected equations from a different, non-local formulation, thereby further justifying the use of this formulation when doing asymptotics. 

In addition, although our goal is to approximate the solution of the water wave problem, the derivation of the KdV equations deserves a special consideration. Here, we obtain the KdV as an equation needed to remove secular terms. In literature, another approach to arrive at KdV is given in \cite{BBM1972}. In the paper, the authors begin by considering a first-order wave equation,
\begin{equation}\label{Model1}
u_t + c_0 u_x = 0,
\end{equation} 
which is a model for small-amplitude, long waves, propagating in $+x$ direction, with speed $c_0.$ The model \eqref{Model1} has limited utility, since the non-linear and dispersive effects accumulate and cause the model to lose its validity over large times.  One can correct for these effects by considering each separately. For non-linearity, this involves approximating the characteristic velocity by making it dependent on $u:$
\[ \frac{1}{c_0} \frac{\D x}{\D t} = 1 +  \epsilon u, \qquad \epsilon \ll 1,\]
so that \eqref{Model1} becomes
\begin{equation}\label{Model2}
u_t + c_0 (1+ \epsilon u)u_x = u_t + c_0 u_x + c_0 \epsilon u u_x  = 0.
\end{equation}
The validity of \eqref{Model2} relies on the condition that the amplitude parameter $\epsilon$ is sufficiently small, and the implicit error is $\mathcal{O}(\epsilon).$ As such, the model \eqref{Model2} can be regarded as an improvement over \eqref{Model1}, accounting for nonlinear effects, within $\mathcal{O}(\epsilon).$ 

Similarly, one can account for the dispersion by considering a linear transformation
\[ Lu = u + \epsilon \alpha^2 u_{xx}, \qquad \epsilon \ll 1.\]
Substituting into \eqref{Model1} yields
\begin{equation}\label{Model3}
u_t + c_0 (Lu)_x= u_t + c_0 u_x + c_0 \epsilon \alpha^2 u_{xxx}  = 0, 
\end{equation}
which can be thought of as an improvement over \eqref{Model1}, accounting for dispersive effects, to $\mathcal{O}(\epsilon).$ 

We obtain \eqref{Model2} and \eqref{Model3} as the respective first-order approximations by allowing for weak nonlinearity and dispersive effects. The authors then argue that an approximation accounting for both effects can be anticipated by simply combining the $\epsilon$ terms:
\begin{equation}\label{Model4}
u_t + u_x + \epsilon(u u_x + \alpha^2 u_{xxx}) = 0,
\end{equation} 
where we set $c_0 =1.$ In nondimensional variables, we transform \eqref{Model4} to obtain 
\[ u_t + u_x + uu_x + u_{xxx} = 0.\] 
Galilean transformations yield the usual form of the KdV 
\[ u_t +uu_x + u_{xxx} = 0. \]

The derivation is indeed elegant, and certainly much shorter, than the one presented here. Note that the use of \eqref{Model1} as the starting point is not problematic: indeed, upon a closer look, one obtains the equation directly from the dynamic boundary condition of the water wave problem, by imposing the shallow water limit. The issue is addition of the $\epsilon$ terms in \eqref{Model2} and \eqref{Model3}: by doing so, the authors already presuppose a certain balance between nonlinearity and dispersion. However, there is no reason to assume this choice of balance; indeed, for a self-consistent theory we must account for the nonlinear and dispersive effects simultaneously. 

\section{Water waves on the half-line}
In the previous section, we use the $\mathcal{H}$ formulation to obtain expected, well-known results. In this section, we use the formulation to study a slightly different problem: the water waves problem, but on the half-line. Physically, we put up a tall, impenetrable barrier at $x = 0.$ This requires imposing several conditions on both $\eta, \phi$ at $x = 0.$ As such, the problem we consider is the following:
\begin{subequations}\label{DimHalfLineProblem}
\begin{align}
\phi_{xx} + \phi_{zz} &= 0, &-h < z < \eta(x,t), \\
\phi_{z} &= 0, &z = -h, \\
\phi_{x} &= 0, &x =0, \label{HLBC1}\\
\eta_t + \phi_{x}\eta_{x} &= \phi_{z}, & z = \eta(x,t), \\
\phi_t + g\eta + \frac{1}{2}(\phi_{x}^2 + \phi_{z}^2) &= 0, &z = \eta(x,t), \\
\phi_{z}(0,\eta,t) &= \eta_t(0,t), &(x,z) = (0,\eta), \label{HLBC2}
\end{align}
\end{subequations}
where $x\in[0,\infty]$ and \eqref{HLBC1}, \eqref{HLBC2} are the new boundary conditions. In particular, \eqref{HLBC1} implies that the fluid does not leak through the barrier at $x=0,$ and \eqref{HLBC2} governs an interaction between the fluid and the surface at $x = 0.$ Approximate equations were conjectured to be the wave and KdV equations. While the wave equation can be justified, there is no reason to expect that we will obtain KdV equations. Indeed, a literature review revealed that KdV has not been derived on the half-line, in the way that we derive the equation on the whole line. 

Using the $\mathcal{H}$ formulation, we derive the approximate equations on the half-line, note the main differences, and discuss the difficulties that arise. To begin, we observe that the scalar equation \eqref{S3:Hequation1} for $\eta$ and $\mathcal{H}$ remains the same, while the non-local equation \eqref{S3:DimHbehav1} changes to:
\begin{equation}\label{S5:DimHLRP1}
\int_{0}^{\infty} \cos(kx) \cosh(k(\eta+h)) f(x) + \sin(kx) \sinh(k(\eta+h))\mathcal{H}(\eta,D) \{f(x) \} \D x = 0,
\end{equation}
and the nondimensional version \eqref{S3:NDimHbehav} becomes
\begin{equation}\label{S5:DimHLRP2}
\int_{0}^{\infty} \cos(kx) \cosh(\mu k(\eta+1)) f(x) + \sin(kx) \sinh(\mu k(\eta+1))\mathcal{H}(\epsilon \eta,D) \{f(x) \} \D x = 0.
\end{equation} 
It is worth noting that taking the real part of the whole-line equations and restricting integrals to $[0,\infty)$ yields the half-line, non-local equations \eqref{DimHLRP1}, \eqref{DimHLRP2}.

By the same procedure, the expansion for $\mathcal{H}$ operator is given by 
\begin{align*}
\mathcal{H}_0(\epsilon\eta, D) \{f(x) \} &= - \{ \mathcal{F}^k_s \}^{-1} \{ \coth(\mu k) \reallywidehat{f^k_c} \}, \\
\mathcal{H}_1(\epsilon\eta, D) \{f(x) \} &= - \{ \mathcal{F}^k_s \}^{-1} \{ \mu k \reallywidehat{\left( \eta f(x) \right)}^k_c + \mu k \coth(\mu k) \reallywidehat{\left( \eta \mathcal{H}_0\{f(x) \} \right)}^k_c\}.
\end{align*}
The notable difference from the whole-line is the presence of Fourier cosine and sine transforms, in place of Fourier transform. 

We apply the expansion to the scalar equation equation. In the leading order, this yields 
\begin{align*}
- \mathcal{F}^k_s \{ \int^x_0 \eta_{tt} \D x' \} + \reallywidehat{\eta_x}^k_s = 0.
\end{align*}
Inverting Fourier sine transform and differentiating with respect to $x$ yields the wave equation on the half-line.

The next order approximation yields the equivalent of \eqref{srfceq0}:
\begin{equation}\label{S5:SurfaceElevationHL}
 \eta_{tt} - \eta_{xx} = \mu^2 \left( \frac{1}{3} \eta_{xxxx}  + \partial_x(\mathcal{F}^k_s)^{-1} \{ \mathcal{F}^k_c \{ \partial_t \left( \eta \int^{x}_0\eta_{t} \D x' \right)\} \} + \frac{1}{2} \partial^2_x\left(\int^{x}_0\eta_{t} \D x' \right)^2\right).
\end{equation}
The notable difference between the equations on two domains is the presence of the inverse sine transform of the cosine transform. Anticipating secularity, we introduce the same time scales 
\[ \tau_0 = t, \qquad \tau_1 = \epsilon t.\]
Along with an expansion $\eta = \eta_0 + \epsilon \eta_1,$ within $\mathcal{O}(\epsilon^0),$ we obtain
\[ \eta_{0\tau_0 \tau_0} - \eta_{0xx} = 0,\]
which is the wave equation on the half line. The general solution on the half-line is 
\[ \eta_0(x, \tau_0, \tau_1, \ldots ) = \begin{cases} F_2(x-\tau_0, \tau_1, \ldots ) + G_2(x+\tau_0, \tau_1, \ldots) & x\geq \tau_0 \\ F_1(\tau_0-x, \tau_1, \ldots ) + G_1(x+\tau_0, \tau_1, \ldots) & x<\tau_0 \end{cases}, \]
where we emphasise that the difference between $F_i$ and $G_i.$ Even though $F_1$ and $F_2$ are both right-going waves, they have different domains, and hence are different functions. Within $\mathcal{O}(\epsilon)$, a careful calculation yields the following system of 4 equations in four unknowns $F_1, F_2, G_1, G_2:$
\begin{equation}\label{S5:HLSystem}
\begin{aligned}
2 \partial_{\tau_1}F_1 &+ \frac{1}{3} \partial_\xi^3 F_1 + (F_1-A)\partial_\xi F_1 \\
&+ \frac{1}{\pi} \left(\int^0_{-\tau_0} (2 F_1  - A)\partial_{\xi'} F_1\frac{1}{\xi -\xi'} \D \xi' + \int^{\infty}_{0} (2 F_2 - (A+B))\partial_{\xi'} F_2 \frac{1}{\xi -\xi'} \D \xi'\right)  &= 0, \qquad \xi < 0; \\
2 \partial_{\tau_1}F_2 &+ \frac{1}{3} \partial_\xi^3 F_2 + (F_2-A-B)\partial_\xi F_2 \\
&+ \frac{1}{\pi}  \left( \int^0_{-\tau_0} (2 F_1 -A)\partial_{\xi'}F_1\frac{1}{\xi -\xi'} \D \xi' + \int^{\infty}_{0} (2 F_2 -  (A+B))\partial_{\xi'} F_2 \frac{1}{\xi -\xi'} \D \xi'  \right) &= 0, \qquad \xi \geq 0; \\
- 2 \partial_{\tau_1} G_1 &+  \frac{1}{3} \partial_\zeta^3 G_1 + (G_1+A) \partial_{\zeta} G_1 \\
&+  \frac{1}{\pi} \left( \int^{2\tau_0}_{\tau_0} (2 G_1 + A) \partial_{\zeta'} G_1 \frac{1}{\zeta -\zeta'} \D \zeta' + \int^{\infty}_{2\tau_0} (2 G_2+(A+B)) \partial_{\zeta'} G_2 \frac{1}{\zeta -\zeta'} \D \zeta' \right) &= 0, \qquad \xi < 0; \\
- 2 \partial_{\tau_1} G_2 &+  \frac{1}{3} \partial_\zeta^3 G_2 + (G_2+ A +B)\partial_{\zeta} G_2 \\
&+ \frac{1}{\pi}  \left( \int^{2\tau_0}_{\tau_0} (2 G_1 +A)\partial_{\zeta'} G_1 \frac{1}{\zeta -\zeta'} \D \zeta' + \int^{\infty}_{2\tau_0} (2 G_2+A+B) \partial_{\zeta'} G_2 \frac{1}{\zeta -\zeta'} \D \zeta' \right) &= 0, \qquad \xi \geq 0,
\end{aligned}
\end{equation}
where 
\[ A = F_1(\tau_0)  - G_1(\tau_0), \qquad B = F_2(0) - F_1(0)  +  G_1(2\tau_0) - G_2(2\tau_0). \]
Each equation in the system \eqref{S5:HLSystem} is somewhat similar to KdV: the time derivative and dispersion term are preserved, whereas nonlinear terms are drastically different. Further, a careful look at the system reveals that unlike on the whole line case, the approximate equations for $F_i$ and $G_i$ are still dependent on the time scale $\tau_0.$ This is an issue, as the reason behind the time scales is to separate the dependence on different time scales. As of now, it is not clear why this issue appears. One possible reason is that the linear time scales
\[ \tau_0 = t, \qquad \tau_1 = \epsilon t\]
should be replaced with different time scales. Another reason could be that that $\mathcal{H}$ formulation does not provide sufficient information to do asymptotics.   

In summary, although we do not obtain the approximate equations on the half-line, we see the utility of the $\mathcal{H}$ formulation in aiding to understand the physical and mathematical difficulties associated with the half-line problem. This should not be taken for granted: for example, if one conducts asymptotic expansions via the velocity potential formulation, one obtains 4 KdV equations for $F_i, G_i, i =1, 2$, which clearly does not agree with the results of this section. As such, the $\mathcal{H}$ formulation shows that the half-line problem has several subtleties, which may not be readily seen in other formulations. 

\bibliographystyle{amsplain}
{\small\bibliography{references}}

\end{document}
