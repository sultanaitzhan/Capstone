\documentclass[10pt,a4paper,final]{article}
\usepackage[left=2cm,right=2cm,top=2cm,bottom=2cm]{geometry}
\input{texHead}
\title{Derivation of Euler's Equations via Chorin \& Marsden}
\thispagestyle{fancy}
\author{Sultan Aitzhan}
\newcommand{\theshorttitle}{Notes from Chorin \& Marsden}
\allowdisplaybreaks

\begin{document}
\maketitle
\thispagestyle{fancy}

\section{Euler's Equations}
In this section, we derive Euler's equations for incompressible, inviscid, and irrotational flows. 

Let $D$ be a region in two or three dimensional space filled with a fluid. We would like to describe the motion of such a fluid. Let $\textbf{x} = (x,y,z) \in D$ be a point in $D$ and consider the particle of fluid moving through $\textbf{x}$ at time $t.$ Let $\textbf{u}(\textbf{x},t)$ denote the velocity of the particle as it moves through $\textbf{x}$ at time $t.$ We call $\textbf{u}$ the \emph{velocity field of the fluid}. 

For each $t,$ suppose there exists a well-defined mass density function $\rho(\textbf{x}, t).$ Then, if $W$ is a subregion of $D,$ the mass of fluid in $W$ at time $t$ is given by 
\[ 
m(W,t) = \int_W \rho(\textbf{x}, t) \D V,
\]
where $dV$ is the volume element in the plane or space. In what follows, we suppose that $\textbf{u}$ and $\rho$ are sufficiently smooth. We derive the equations of motion based on three main principles:
\begin{enumerate}
\item mass is neither created nor destroyed;
\item the rate of change of momentum in a portion of the fluid equals the force applied to it;
\item energy is neither created nor destroyed.
\end{enumerate}

\subsection{Conservation of Mass}

Let $W$ be a subregion of $D.$ Then, the rate of change of mass in W is 
\begin{equation}\label{CoM:eq1}
\frac{d}{dt} m(W,t) = \frac{d}{dt} \int_W \rho(\textbf{x}, t) \D V =  \int_W \frac{\partial}{\partial t} \rho(\textbf{x}, t) \D V. 
\end{equation}

Let $\partial W$ denote the boundary of $W,$ let ${\bf n}$ denote the unit outward normal defined on points of $\partial W,$ and let $\D A$ denote the area element on $\partial W.$ Note that the volume flow rate, i.e. the volume of fluid which passes across $\partial W$ per unit time, per unit area is ${\bf u} \cdot {\bf n},$ and the mass flow rate, i.e. the mass of substance that passes across $\partial W$ per unit time, per unit area is $\rho {\bf u}\dot {\bf n}.$ In particular,  
\begin{equation}\label{CoM:eq2}
\mbox{the mass crossing the boundary $\partial W$ per unit time} = \int_{\partial W} \rho {\bf u} \cdot {\bf n} \D A.
\end{equation}

The principle of conservation of mass can be stated in the following way: the rate of increase of mass in $W$ equals the rate at which mass is crossing $\partial W$ in the inward direction. Using \eqref{CoM:eq1} and \eqref{CoM:eq2}, we have 
\begin{equation}\label{CoM:eq3}
\frac{d}{dt} \int_W \rho(\textbf{x}, t) \D V = - \int_{\partial W} \rho \textbf{u} \cdot \textbf{n} \D A.
\end{equation}
This is the integral form of conservation of mass principle. 
In addition, we have the differential form of this principle, obtained as follows:
\begin{align}
\frac{d}{dt} \int_W \rho({\bf x}, t) \D V = \int_W \frac{\partial}{\partial t} \rho({\bf x}, t) \D V &\implies  \int_W \frac{\partial}{\partial t} \rho({\bf x}, t) \D V  = - \int_{\partial W} \rho {\bf u} \cdot {\bf n} \D A \nonumber \\
&\implies \int_W \frac{\partial}{\partial t} \rho({\bf x}, t) \D V  + \int_{\partial W} \rho {\bf u} \cdot {\bf n} \D A = 0 \nonumber \\
&\implies \int_W \frac{\partial}{\partial t} \rho({\bf x}, t) \D V + \int_W {\rm div} \rho {\bf u} \D V = 0 \nonumber \\
&\implies \int_W \frac{\partial}{\partial t} \rho({\bf x}, t) +  {\rm div} \rho {\bf u} \D V = 0 \label{CoM:eq4} \\
&\implies \frac{\partial}{\partial t} \rho({\bf x}, t) +  {\rm div} \rho {\bf u} = \frac{D\rho}{Dt} = 0
\end{align} 
which is obtained by applying the divergence theorem, noting that \eqref{CoM:eq4} must hold for any subregion $W$ and defining $\displaystyle\frac{D}{Dt} = \partial_t + \U\cdot \nabla$ is the \emph{material derivative}.

\subsection{Balance of Momentum}
Let ${\bf x}(t) = (x(t), y(t), z(t))$ be the trajectory followed by a fluid particle, so that the velocity is given by 
\[ 
{\bf u}(x(t), y(t), z(t), t) = (x\dot(t), y\dot(t), z\dot(t)) = \frac{d {\bf x} }{d t}(t).
\]
The acceleration of a fluid particle is given by 
\[ {\bf a}(t) = \frac{d^2}{dt^2}{\bf x}(t) = \frac{d}{dt} {\bf u}(x(t), y(t), z(t), t),\]
which becomes 
\[ 
{\bf a}(t) = \frac{\partial {\bf u}}{\partial x}\dot{x} + \frac{\partial {\bf u}}{\partial y}\dot{y} + \frac{\partial {\bf u}}{\partial z}\dot{z} + \frac{\partial {\bf u}}{\partial t}
\]
by chain rule. Letting
\[ 
{\bf u}(x,y,z,t) = (u(x,y,z,t), v(x,y,z,t), w(x,y,z,t))
\]
implies $u = \dot{x}, v = \dot{y}, w = \dot{z},$ so that we have 
\[ 
{\bf a}(t) = u\frac{\partial {\bf u}}{\partial x} + u\frac{\partial {\bf u}}{\partial y}+ z\frac{\partial {\bf u}}{\partial z} + \frac{\partial {\bf u}}{\partial t} = \partial_t {\bf u} + {\bf u} \cdot \nabla {\bf u}.
\]
We call the function $\frac{D}{Dt} = \partial_t + {\bf u} \cdot \nabla$ the \emph{material derivative}; it accounts for the motion of fluid and changes in the positions of fluid particles with time.

Now, for any continuum, there are two types of forces acting on any piece of material. One is forces of \emph{stress}, where a piece of material on a given continuum is acted on by forces from the rest of continuum. Second, there are external forces such as body, magnetic fields, or gravity. 

We start off with an \emph{ideal fluid}, which has the following property: for any motion of fluid there is a function $p(\X, t)$ called \emph{pressure} such that if $S$ is surface in the fluid with a chosen unit normal $\N,$ the force of stress exerted on $S$ per unit area at $\X \in S$ at time $t$ is $p(\X,t).$ This assumption suggests that the fluid is inviscid. If $W$ is a region in the fluid at time $t,$ the total force exerted on the fluid inside $W$ by means of stress on its boundary is 
\[ 
{\bf S}_{\partial W} = \{ \mbox{force on W}\} = - \int_{\partial W} p\N \D A.
\]
For any vector ${\bf e},$ we have 
\begin{align*}
{\bf e} \cdot {\bf S}_{\partial W} = - \int_{\partial W} p{\bf e} \cdot \N \D A &= - \int_{W} {\rm div}p{\bf e}\D V \\
&= - \int_{W} p{\rm div}{\bf e} + (\nabla p) \cdot {\bf e} \D V \\
&= - \int_{W} ({\rm grad} p) \cdot {\bf e} \D V, 
\end{align*}
so that 
\begin{equation}\label{BoM:eq1}
{\bf S}_{\partial W} = - \int_{W} {\rm grad} p \D V.
\end{equation} 
If ${\bf b}(\X,t)$ denotes the given body force per unit mass, the total body force is given 
\begin{equation}\label{BoM:eq2}
{\bf B} = \int_{W} \rho{\bf b} \D V.
\end{equation}
Thus, combining \eqref{BoM:eq1} and \eqref{BoM:eq2} will yield that on any piece of fluid material,
\[ 
\mbox{force per unit volume} = - {\rm grad} p + \rho{\bf b}.
\]
Now, recall that $\rho$ indicates mass and $\frac{D\U}{Dt}$ indicates acceleration. By Newton's Second law ($F = ma$), we can conclude that
\begin{equation}\label{BoM:eq3}
\rho\frac{D\U}{Dt} = - {\rm grad} p + \rho{\bf b}.
\end{equation}
This is the differential form of the principle of balance of momentum. It can be shown that the integral form of the principle is 
\begin{equation}\label{BoM:eq4}
\int_{W} \rho {\bf u}  \D V = - \int_{\partial W} (p\N + \rho \U(\U \cdot \N)) \D A +\int_{W} \rho{\bf b} \D V.
\end{equation}
\subsection{Conservation of Energy}
We develop the final equation. For a fluid moving in a domain $D$ with velocity field $\U,$ the \emph{kinetic energy} contained in a region $W \subset D$ is 
\[ 
E_{\rm kinetic} = \frac{1}{2} \int_W \rho \| \U\|^2 \D V,
\]
where $\| \U\|^2 = (u^2 + v^2 + w^2).$ Assume that we can write $E_{\rm total} = E_{\rm kinetic} + E_{\rm internal}.$ We proceed to find an expression for $E_{\rm kinetic}.$

The rate of change of kinetic energy of a moving portion $W_t$ of fluid is calculated with the aid of the transport theorem, whereby we have 
\[ 
\frac{d}{dt}\int_{W_t} \rho f \D V = \int_{W_t} \rho \frac{Df}{Dt} \D V.
\]
Then, 
\begin{align}
\frac{d}{dt} E_{\rm kinetic}  &=  \frac{1}{2} \frac{d}{dt} \int_{W_t} \rho \| \U\|^2 \D V \\
&= \frac{1}{2}  \int_{W_t} \rho \frac{D \| \U\|^2}{Dt} \D V \\
&= \frac{1}{2} \int_{W_t} 2 \rho  \U \cdot \frac{D \U}{Dt} \D V \\
&= \int_{W_t} \rho  \U \cdot \frac{D \U}{Dt} \D V \\
&= \int_{W_t} \rho  \U \cdot \left( \frac{\partial \U}{\partial t} + (\U \cdot \nabla) \U \right) \D V.
\end{align}

Now, suppose that all the energy is kinetic (i.e. that the fluid is \emph{incompressible}) and that the rate of change of kinetic energy in a portion of fluid equals the rate at which the pressure and body forces do work. In other words, assume
\[ 
\frac{d}{dt} E_{\rm kinetic} = - \int_{\partial W_t} p \U \cdot \N \D a + \int_{W_t} \rho \U \cdot {\bf b} \D V.
\]
This assumption yields the following:
\begin{align*}
\int_{W_t} \rho  \U \cdot \left( \frac{\partial \U}{\partial t} + (\U \cdot \nabla) \U \right) \D V &= \frac{d}{dt} E_{\rm kinetic} \\
&= - \int_{\partial W_t} p \U \cdot \N \D A + \int_{W_t} \rho \U \cdot {\bf b} \D V &\text{(by assumption)}\\
&= - \int_{W_t} {\rm div}\ p \U - \rho \U \cdot {\bf b} \D V &\text{(by divergence theorem)} \\
&= - \int_{W_t} \U \cdot \nabla p - \rho \U \cdot {\bf b} \D V &\text{(since ${\rm div}\ \U = 0$)}.
\end{align*}
Then, for incompressible fluids, the \emph{Euler equations} are 
\begin{align*}
\rho\frac{D\U}{Dt} &= - {\rm grad} p + \rho{\bf b} &\text{Balance of momentum}\\ 
\frac{D \rho}{D t} &= 0 &\text{Conservation of mass} \\
{\rm div}\ \U &= 0 &\text{Fluid is incompressible}
\end{align*}
along with the boundary condition 
\[ \U \cdot \N = 0 \quad \mbox{on } \partial D.\]
\end{document}