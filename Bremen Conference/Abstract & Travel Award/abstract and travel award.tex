% Preamble
\documentclass[10pt,reqno,oneside,a4paper]{article}
\usepackage[a4paper,includeheadfoot,left=25mm,right=25mm,top=00mm,bottom=20mm,headheight=20mm]{geometry}
\usepackage{siunitx}
\input{texHead}
\author{Sultan Aitzhan}
\rhead{Draft 2}
\newcommand{\theshorttitle}{}
\date{\today}
\allowdisplaybreaks

\begin{document}
\thispagestyle{fancy}

\subsection*{Application for a travel award}
My name is Sultan, and I am a final year student at Yale-NUS College in Singapore. I am writing to request a SIAM Student Travel award, which will allow me to attend the SIAM NWCS20 conference, present my research, and help me advance mathematics. 

Attending this conference comes at an important moment as I will be starting my PhD in mathematics. 
One reason for this is that I will get an opportunity to present a poster, titled \textit{“Approximate equations for the water wave problem in the shallow water limit on the whole and half lines"}, which is based on the research work I conducted under the supervision of Katie Oliveras. The poster presents a derivation of shallow water, approximate equations from a certain non-local formulation, alongside the relevant numerical work. Describing my findings to and getting feedback from conference participants will enable me to refine my work and possibly lead to new directions. This is one way in which NWCS20 will help me develop me as a mathematician.

In addition, as I mature and become aware of many fields of mathematics as well as its applications, the conference will allow me to learn of general trends in the field of nonlinear waves and coherent structures. Indeed, nonlinear waves is one of my research interests, and this conference has featured many minisymposia on the numerous aspects of this topic. In particular, I am interested in how the methods of spectral theory, scientific computing, and asymptotic analysis are used in studying and resolving problems of nonlinear waves. To this end, attending plenary talks by many experts such as Bernard Deconinck, Sergey Nazarenko, and David Henry as well as minisymposia will allow me to attain this goal. As such, conference talks and conversations with speakers will provide me with valuable perspectives on the subject, and in turn, motivate the development of my own work in this field. 

Finally, I am applying for the student travel award because at the time of the conference, I will have graduated from my undergraduate institution, but have yet to start graduate school. This leaves me in a position of being unable to apply for any funding at either my college or my graduate school. Thus, the travel award will give me a chance to enjoy the conference during this inconvenient transition. 

Altogether, sharing my research and becoming better acquainted with the field of nonlinear waves and coherent structures will kickstart my journey as a mathematician and help me advance the field of mathematics. Thank you very much for your consideration.

Sincerely,

Sultan Aitzhan

\subsection*{Poster abstract}
Poster title: \textbf{Approximate equations for the water wave problem in the shallow water limit on the whole and half lines}

A free boundary, water wave problem is studied for an irrotational, inviscid, and incompressible fluid. Specifically, we describe a derivation of approximate equations in the shallow water limit using a non-local formulation, introduced in \cite{KV2013} via a normal-to-tangential operator, in two related settings. One is the classical, whole line case, and another is a half line case, which physically is represented by putting up a tall and impenetrable barrier in the middle, so that all the fluid is flowing to one side. In both settings, non-local formulations yield expressions for the surface elevation, from which the appropriate wave and KdV equations are obtained. We present a numerical algorithm using the non-local formulation, and the utility of a normal-to-tangential operator is examined via its numerical error, along with comparison to DNO \cite{DNO} and AFM \cite{AFM} formulations. Connections between the two settings are explored, and a number of interesting differences are noted.

\bibliographystyle{amsplain}
{\small\bibliography{references}}

\end{document}
