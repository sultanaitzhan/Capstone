% Preamble
\documentclass[10pt,reqno,oneside,a4paper]{article}
\usepackage[a4paper,includeheadfoot,left=25mm,right=25mm,top=00mm,bottom=20mm,headheight=20mm]{geometry}
\usepackage{siunitx}
\input{texHead}
\author{Sultan Aitzhan}
\newcommand{\theshorttitle}{}
\date{\today}
\allowdisplaybreaks

\begin{document}
\thispagestyle{fancy}

\textbf{Poster title}: \textit{Approximate equations for the water wave problem in the shallow water limit on unbounded domains}

\textbf{Poster abstract} : A free boundary, water wave problem is studied for an irrotational, inviscid, and incompressible fluid. Specifically, we describe a derivation of approximate equations in the shallow water limit using a non-local formulation, introduced in \cite{KV2013} via a normal-to-tangential operator, in two related settings. One is the classical, whole line case, and another is a half line case, which physically is represented by putting up a tall and impenetrable barrier in the middle, so that all the fluid is flowing to one side. In both settings, non-local formulations yield expressions for the surface elevation, from which the appropriate wave and Boussinesq equations, as well as new approximations are obtained. We present a numerical algorithm using the non-local formulation, and the utility of a normal-to-tangential operator is examined via its numerical error, along with comparison to DNO \cite{DNO} and AFM \cite{AFM} formulations. Connections between the two settings are explored, and a number of interesting differences are noted.

\bibliographystyle{amsplain}
{\small\bibliography{references}}

\end{document}
